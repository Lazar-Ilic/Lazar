% Perhaps Typing In Some Formulae Here Will Look OK From The Outside Like I Care, Which I Do


Contents



Preface ....................................................... V



General notations . . . . . . . . . . . . . . . . . . . . . . . . . . . . . . . . . . . . . . . . . . . . . XV



1 Derivatives and arbitrage pricing ......................... 1



1.1 Options . . . . . . . . . . . . . . . . . . . . . . . . . . . . . . . . . . . . . . . . . . . . . . . 1



1.1.1 Main purposes . . . . . . . . . . . . . . . . . . . . . . . . . . . . . . . . . . . 3



1.1.2 Main problems . . . . . . . . . . . . . . . . . . . . . . . . . . . . . . . . . . . 4



1.1.3 Rules of compounding ............................ 4



1.1.4 Arbitrage opportunities and Put-Call parity formula . . 5

OKOKOK. I dunno if this textbook is useful for thinking about what might appear in the Citadel Securities codebase.

Replicating Condition

Arbitrage Price

1.2 Risk-neutral price and arbitrage pricing . . . . . . . . . . . . . . . . . . . 7



1.2.1 Risk-neutral price . . . . . . . . . . . . . . . . . . . . . . . . . . . . . . . . 7



1.2.2 Risk-neutral probability ........................... 8



1.2.3 Arbitrage price . . . . . . . . . . . . . . . . . . . . . . . . . . . . . . . . . . . 8



1.2.4 A generalization of the Put-Call parity . . . . . . . . . . . . . . 10



1.2.5 Incomplete markets . . . . . . . . . . . . . . . . . . . . . . . . . . . . . . . 11



2 Discrete market models . . . . . . . . . . . . . . . . . . . . . . . . . . . . . . . . . . . 15

Interesting enough citations.

Self-Financing

2.1 Discrete markets and arbitrage strategies . . . . . . . . . . . . . . . . . . 15



2.1.1 Self-financing and predictable strategies . . . . . . . . . . . . . 16



2.1.2 Normalized market . . . . . . . . . . . . . . . . . . . . . . . . . . . . . . . 19



2.1.3 Arbitrage opportunities and admissible strategies . . . . . 20



2.1.4 Equivalent martingale measure . . . . . . . . . . . . . . . . . . . . . 21



2.1.5 Change of numeraire . . . . . . . . . . . . . . . . . . . . . . . . . . . . . . 24



2.2 European derivatives . . . . . . . . . . . . . . . . . . . . . . . . . . . . . . . . . . . . 26



2.2.1 Pricing in an arbitrage-free market . . . . . . . . . . . . . . . . . 27



2.2.2 Completeness . . . . . . . . . . . . . . . . . . . . . . . . . . . . . . . . . . . . 30



2.2.3 Fundamental theorems of asset pricing . . . . . . . . . . . . . . 31



2.2.4 Markov property . . . . . . . . . . . . . . . . . . . . . . . . . . . . . . . . . 34



2.3 Binomial model . . . . . . . . . . . . . . . . . . . . . . . . . . . . . . . . . . . . . . . . 35



2.3.1 Martingale measure and arbitrage price . . . . . . . . . . . . . 38



2.3.2 Hedging strategies . . . . . . . . . . . . . . . . . . . . . . . . . . . . . . . . 40



VIII Contents



2.3.3 Binomial algorithm . . . . . . . . . . . . . . . . . . . . . . . . . . . . . . . 45

This part is pretty canonical in These Things Also Were Heard On The Street.

2.3.4 Calibration . . . . . . . . . . . . . . . . . . . . . . . . . . . . . . . . . . . . . . 50



2.3.5 Binomial model and Black-Scholes formula . . . . . . . . . . 53

OK

2.3.6 Black-Scholes differential equation . . . . . . . . . . . . . . . . . . 60



2.4 Trinomial model. . . . . . . . . . . . . . . . . . . . . . . . . . . . . . . . . . . . . . . . 62

OK.

2.4.1 Pricing and hedging in an incomplete market . . . . . . . . 66

OK

2.5 American derivatives . . . . . . . . . . . . . . . . . . . . . . . . . . . . . . . . . . . . 72



2.5.1 Arbitrage price . . . . . . . . . . . . . . . . . . . . . . . . . . . . . . . . . . . 74

OK

2.5.2 Optimal exercise strategies . . . . . . . . . . . . . . . . . . . . . . . . 80

OK

2.5.3 Pricing and hedging algorithms . . . . . . . . . . . . . . . . . . . . 83



2.5.4 Relations with European options . . . . . . . . . . . . . . . . . . . 88

OK

2.5.5 Free-boundary problem for American options . . . . . . . . 90

OK

2.5.6 American and European options in the binomial model 93



3 Continuous-time stochastic processes . . . . . . . . . . . . . . . . . . . . . 97

OK

3.1 Stochastic processes and real Brownian motion . . . . . . . . . . . . . 97



3.1.1 Markov property . . . . . . . . . . . . . . . . . . . . . . . . . . . . . . . . . 100

OK

3.1.2 Brownian motion and the heat equation . . . . . . . . . . . . . 102

OK think maybe memorising this book for interviews is alright.

3.2 Uniqueness . . . . . . . . . . . . . . . . . . . . . . . . . . . . . . . . . . . . . . . . . . . . 103



3.2.1 Law of a continuous process . . . . . . . . . . . . . . . . . . . . . . . 103

OK

3.2.2 Equivalence of processes . . . . . . . . . . . . . . . . . . . . . . . . . . . 105



3.2.3 Modifications and indistinguishable processes . . . . . . . . 107

OK

3.2.4 Adapted and progressively measurable processes. . . . . . 110



3.3 Martingales . . . . . . . . . . . . . . . . . . . . . . . . . . . . . . . . . . . . . . . . . . . . 111

OK

3.3.1 Doob's inequality . . . . . . . . . . . . . . . . . . . . . . . . . . . . . . . . . 113



3.3.2 Martingale spaces: M2 and M2

OK

c . . . . . . . . . . . . . . . . . . . . 114



3.3.3 The usual hypotheses . . . . . . . . . . . . . . . . . . . . . . . . . . . . . 117

OK

3.3.4 Stopping times and martingales . . . . . . . . . . . . . . . . . . . . 120

OK

3.4 Riemann-Stieltjes integral . . . . . . . . . . . . . . . . . . . . . . . . . . . . . . . 125

OK

3.4.1 Bounded-variation functions . . . . . . . . . . . . . . . . . . . . . . . 127

OK

3.4.2 Riemann-Stieltjes integral and It'o formula . . . . . . . . . . . 131

OK

3.4.3 Regularity of the paths of a Brownian motion . . . . . . . . 134



4 Brownian integration . . . . . . . . . . . . . . . . . . . . . . . . . . . . . . . . . . . . . 139

OK

4.1 Stochastic integral of deterministic functions . . . . . . . . . . . . . . . 140



4.2 Stochastic integral of simple processes . . . . . . . . . . . . . . . . . . . . . 141

OK

4.3 Integral of L2-processes . . . . . . . . . . . . . . . . . . . . . . . . . . . . . . . . . 145

OK

4.3.1 Ito and Riemann-Stieltjes integral . . . . . . . . . . . . . . . . . . 149

OK

4.3.2 Ito integral and stopping times . . . . . . . . . . . . . . . . . . . . . 151

OK

4.3.3 Quadratic variation process . . . . . . . . . . . . . . . . . . . . . . . . 153

OK

4.3.4 Martingales with bounded variation . . . . . . . . . . . . . . . . 156



4.3.5 Co-variation process . . . . . . . . . . . . . . . . . . . . . . . . . . . . . . 157

OK

4.4 Integral of L2

OK

loc-processes . . . . . . . . . . . . . . . . . . . . . . . . . . . . . . . . 159



4.4.1 Local martingales . . . . . . . . . . . . . . . . . . . . . . . . . . . . . . . . 161



4.4.2 Localization and quadratic variation . . . . . . . . . . . . . . . . 163

OK

Contents IX



5 Ito calculus . . . . . . . . . . . . . . . . . . . . . . . . . . . . . . . . . . . . . . . . . . . . . . . 167



5.1 Ito processes . . . . . . . . . . . . . . . . . . . . . . . . . . . . . . . . . . . . . . . . . . . 168



5.1.1 Ito formula for Brownian motion . . . . . . . . . . . . . . . . . . . 169



5.1.2 General formulation . . . . . . . . . . . . . . . . . . . . . . . . . . . . . . 174



5.1.3 Martingales+and parabolic equations . . . . . . . . . . . . . . . 176



5.1.4 Geometric Brownian motion . . . . . . . . . . . . . . . . . . . . . . . 176



5.2 Multi-dimensional Ito processes . . . . . . . . . . . . . . . . . . . . . . . . . . 179



5.2.1 Multi-dimensional Ito formula. . . . . . . . . . . . . . . . . . . . . . 183



5.2.2 Correlated Brownian motion+and martingales . . . . . . . 188



5.3 Generalized Ito formulas . . . . . . . . . . . . . . . . . . . . . . . . . . . . . . . . 191



5.3.1 Ito formula and+weak derivatives . . . . . . . . . . . . . . . . . . 191



5.3.2 Tanaka formula+and local times . . . . . . . . . . . . . . . . . . . 194



5.3.3 Tanaka+formula for Ito processes . . . . . . . . . . . . . . . . . . 197

OK

5.3.4 Local+time and Black-Scholes formula . . . . . . . . . . . . . . 198



6 Parabolic PDEs with variable coefficients: uniqueness . . . . . 203



6.1 Maximum principle and Cauchy-Dirichlet problem . . . . . . . . . . 206

OK

6.2 Maximum principle and Cauchy problem . . . . . . . . . . . . . . . . . . 208

OK

6.3 Non-negative solutions of the Cauchy problem . . . . . . . . . . . . . 213



7 Black-Scholes model . . . . . . . . . . . . . . . . . . . . . . . . . . . . . . . . . . . . . . 219

OK

7.1 Self-financing strategies . . . . . . . . . . . . . . . . . . . . . . . . . . . . . . . . . 220

OK

7.2 Markovian strategies and Black-Scholes equation . . . . . . . . . . . 222

OK

7.3 Pricing . . . . . . . . . . . . . . . . . . . . . . . . . . . . . . . . . . . . . . . . . . . . . . . . 225



7.3.1 Dividends and time-dependent parameters . . . . . . . . . . . 228

OK

7.3.2 Admissibility and absence of arbitrage . . . . . . . . . . . . . . 229

OK

7.3.3 Black-Scholes analysis: heuristic approaches . . . . . . . . . . 231



7.3.4 Market price of risk . . . . . . . . . . . . . . . . . . . . . . . . . . . . . . . 233



7.4 Hedging . . . . . . . . . . . . . . . . . . . . . . . . . . . . . . . . . . . . . . . . . . . . . . . 236

OK

7.4.1 The Greeks . . . . . . . . . . . . . . . . . . . . . . . . . . . . . . . . . . . . . . 236



7.4.2 Robustness of the model . . . . . . . . . . . . . . . . . . . . . . . . . . 245

OK

7.4.3 Gamma and Vega-hedging . . . . . . . . . . . . . . . . . . . . . . . . . 246

OK

7.5 Implied volatility . . . . . . . . . . . . . . . . . . . . . . . . . . . . . . . . . . . . . . . 248



7.6 Asian options . . . . . . . . . . . . . . . . . . . . . . . . . . . . . . . . . . . . . . . . . . 252



7.6.1 Arithmetic average . . . . . . . . . . . . . . . . . . . . . . . . . . . . . . . 253



7.6.2 Geometric average . . . . . . . . . . . . . . . . . . . . . . . . . . . . . . . . 255

OK

8 Parabolic PDEs with variable coefficients: existence . . . . . . 257



8.1 Cauchy problem and fundamental solution . . . . . . . . . . . . . . . . . 258

OK

8.1.1 Levi's parametrix method . . . . . . . . . . . . . . . . . . . . . . . . . 260

OK

8.1.2 Gaussian estimates and adjoint operator . . . . . . . . . . . . 261



8.2 Obstacle problem . . . . . . . . . . . . . . . . . . . . . . . . . . . . . . . . . . . . . . . 263



8.2.1 Strong solutions . . . . . . . . . . . . . . . . . . . . . . . . . . . . . . . . . . 265



8.2.2 Penalization method . . . . . . . . . . . . . . . . . . . . . . . . . . . . . . 268

OK

X Contents



9 Stochastic differential equations . . . . . . . . . . . . . . . . . . . . . . . . . . 275



9.1 Strong solutions . . . . . . . . . . . . . . . . . . . . . . . . . . . . . . . . . . . . . . . . 276

OK

9.1.1 Uniqueness . . . . . . . . . . . . . . . . . . . . . . . . . . . . . . . . . . . . . . 278



9.1.2 Existence . . . . . . . . . . . . . . . . . . . . . . . . . . . . . . . . . . . . . . . . 280



9.1.3 Properties of solutions . . . . . . . . . . . . . . . . . . . . . . . . . . . . 283



9.2 Weak solutions . . . . . . . . . . . . . . . . . . . . . . . . . . . . . . . . . . . . . . . . . 286



9.2.1 Tanaka's example . . . . . . . . . . . . . . . . . . . . . . . . . . . . . . . . 286

OK

9.2.2 Existence: the martingale problem . . . . . . . . . . . . . . . . . . 287



9.2.3 Uniqueness . . . . . . . . . . . . . . . . . . . . . . . . . . . . . . . . . . . . . . 290

OK

9.3 Maximal estimates. . . . . . . . . . . . . . . . . . . . . . . . . . . . . . . . . . . . . . 292



9.3.1 Maximal estimates for martingales . . . . . . . . . . . . . . . . . . 293

OK

9.3.2 Maximal estimates for diffusions. . . . . . . . . . . . . . . . . . . . 296



9.4 Feynman-Kac representation formulas . . . . . . . . . . . . . . . . . . . . . 298



9.4.1 Exit time from a bounded domain . . . . . . . . . . . . . . . . . . 300

OK

9.4.2 Elliptic-parabolic equations and Dirichlet problem . . . . 302



9.4.3 Evolution equations and Cauchy-Dirichlet problem . . . 307

OK

9.4.4 Fundamental solution and transition density . . . . . . . . . 308



9.4.5 Obstacle problem and optimal stopping . . . . . . . . . . . . . 310

OK

9.5 Linear equations . . . . . . . . . . . . . . . . . . . . . . . . . . . . . . . . . . . . . . . 314

OK

9.5.1 Kalman condition . . . . . . . . . . . . . . . . . . . . . . . . . . . . . . . . 318



9.5.2 Kolmogorov equations and Hormander condition . . . . . 323

OK

9.5.3 Examples . . . . . . . . . . . . . . . . . . . . . . . . . . . . . . . . . . . . . . . 326



10 Continuous market models . . . . . . . . . . . . . . . . . . . . . . . . . . . . . . . 329

OK

10.1 Change of measure . . . . . . . . . . . . . . . . . . . . . . . . . . . . . . . . . . . . . 329



10.1.1 Exponential martingales . . . . . . . . . . . . . . . . . . . . . . . . . . . 329

OK

10.1.2 Girsanov's theorem . . . . . . . . . . . . . . . . . . . . . . . . . . . . . . . 332

OK

10.1.3 Representation of Brownian martingales . . . . . . . . . . . . . 334



10.1.4 Change of drift . . . . . . . . . . . . . . . . . . . . . . . . . . . . . . . . . . . 339



10.2 Arbitrage theory . . . . . . . . . . . . . . . . . . . . . . . . . . . . . . . . . . . . . . . 340

OK

10.2.1 Change of drift with correlation . . . . . . . . . . . . . . . . . . . . 343

OK

10.2.2 Martingale measures and market prices of risk . . . . . . . 345



10.2.3 Examples . . . . . . . . . . . . . . . . . . . . . . . . . . . . . . . . . . . . . . . 348



10.2.4 Admissible strategies and arbitrage opportunities . . . . . 352



10.2.5 Arbitrage pricing . . . . . . . . . . . . . . . . . . . . . . . . . . . . . . . . . 355

OK

10.2.6 Complete markets . . . . . . . . . . . . . . . . . . . . . . . . . . . . . . . . 357



10.2.7 Parity formulas . . . . . . . . . . . . . . . . . . . . . . . . . . . . . . . . . . 358

OK

10.3 Markovian models: the PDE approach . . . . . . . . . . . . . . . . . . . . 359

OK

10.3.1 Martingale models for the short rate . . . . . . . . . . . . . . . . 361

OK

10.3.2 Pricing and hedging in a complete model . . . . . . . . . . . . 364



10.4 Change of numeraire . . . . . . . . . . . . . . . . . . . . . . . . . . . . . . . . . . . . 366

OK

10.4.1 LIBOR market model . . . . . . . . . . . . . . . . . . . . . . . . . . . . . 370



10.4.2 Change of numeraire for Ito processes . . . . . . . . . . . . . . . 372

OK

10.4.3 Pricing with stochastic interest rate. . . . . . . . . . . . . . . . . 374

OK

10.5 Diffusion-based volatility models . . . . . . . . . . . . . . . . . . . . . . . . . 376

OK

Contents XI



10.5.1 Local and path-dependent volatility . . . . . . . . . . . . . . . . 377

OK

10.5.2 CEV model . . . . . . . . . . . . . . . . . . . . . . . . . . . . . . . . . . . . . . 379

OK

10.5.3 Stochastic volatility and the SABR model . . . . . . . . . . . 386



11 American options . . . . . . . . . . . . . . . . . . . . . . . . . . . . . . . . . . . . . . . . . 389

OK

11.1 Pricing and hedging in the Black-Scholes model . . . . . . . . . . . . 389



11.2 American Call and Put options in the Black-Scholes model . . 395

OK

11.3 Pricing and hedging in a complete market . . . . . . . . . . . . . . . . . 398



12 Numerical methods . . . . . . . . . . . . . . . . . . . . . . . . . . . . . . . . . . . . . . . 403

OK

12.1 Euler method for ordinary equations . . . . . . . . . . . . . . . . . . . . . . 403

OK

12.1.1 Higher order schemes . . . . . . . . . . . . . . . . . . . . . . . . . . . . . 407

OK

12.2 Euler method for stochastic differential equations . . . . . . . . . . . 408



12.2.1 Milstein scheme . . . . . . . . . . . . . . . . . . . . . . . . . . . . . . . . . . 411



12.3 Finite-difference methods for parabolic equations . . . . . . . . . . . 412

OK

12.3.1 Localization . . . . . . . . . . . . . . . . . . . . . . . . . . . . . . . . . . . . . 413



12.3.2 θ-schemes for the Cauchy-Dirichlet problem . . . . . . . . . . 414

OK

12.3.3 Free-boundary problem . . . . . . . . . . . . . . . . . . . . . . . . . . . 419

OK

12.4 Monte Carlo methods . . . . . . . . . . . . . . . . . . . . . . . . . . . . . . . . . . . 420



12.4.1 Simulation. . . . . . . . . . . . . . . . . . . . . . . . . . . . . . . . . . . . . . . 423



12.4.2 Computation of the Greeks . . . . . . . . . . . . . . . . . . . . . . . . 425

OKOKOK

12.4.3 Error analysis . . . . . . . . . . . . . . . . . . . . . . . . . . . . . . . . . . . . 427



13 Introduction to Levy processes . . . . . . . . . . . . . . . . . . . . . . . . . . . 429

OK

13.1 Beyond Brownian motion . . . . . . . . . . . . . . . . . . . . . . . . . . . . . . . . 429



13.2 Poisson process . . . . . . . . . . . . . . . . . . . . . . . . . . . . . . . . . . . . . . . . 432

OK

13.3 Levy processes . . . . . . . . . . . . . . . . . . . . . . . . . . . . . . . . . . . . . . . . . 437



13.3.1 Infinite divisibility and characteristic function . . . . . . . . 439

OK

13.3.2 Jump measures of compound Poisson processes. . . . . . . 444

OK

13.3.3 Levy-Ito decomposition . . . . . . . . . . . . . . . . . . . . . . . . . . . 450

OK

13.3.4 Levy-Khintchine representation . . . . . . . . . . . . . . . . . . . . 457

OK

13.3.5 Cumulants and Levy martingales . . . . . . . . . . . . . . . . . . . 460

OK

13.4 Examples of Levy processes . . . . . . . . . . . . . . . . . . . . . . . . . . . . . . 463

OK

13.4.1 Jump-diffusion processes . . . . . . . . . . . . . . . . . . . . . . . . . . 464

OKOKOK need to pay attention to this part too.

13.4.2 Stable processes . . . . . . . . . . . . . . . . . . . . . . . . . . . . . . . . . . 466

OK

13.4.3 Tempered stable processes . . . . . . . . . . . . . . . . . . . . . . . . . 469

OK

13.4.4 Subordination. . . . . . . . . . . . . . . . . . . . . . . . . . . . . . . . . . . . 471

OK

13.4.5 Hyperbolic processes . . . . . . . . . . . . . . . . . . . . . . . . . . . . . . 478

OK

13.5 Option pricing under exponential Levy processes . . . . . . . . . . . 480

OK

13.5.1 Martingale modeling in Levy markets . . . . . . . . . . . . . . . 480

OK

13.5.2 Incompleteness and choice of an EMM . . . . . . . . . . . . . . 485

OK

13.5.3 Esscher transform . . . . . . . . . . . . . . . . . . . . . . . . . . . . . . . . 486

OK

13.5.4 Exotic option pricing . . . . . . . . . . . . . . . . . . . . . . . . . . . . . 491

OK

13.5.5 Beyond Levy processes . . . . . . . . . . . . . . . . . . . . . . . . . . . . 494

OK

XII Contents



14 Stochastic calculus for jump processes . . . . . . . . . . . . . . . . . . . . 497

OK

14.1 Stochastic integrals . . . . . . . . . . . . . . . . . . . . . . . . . . . . . . . . . . . . . 497

OK

14.1.1 Predictable processes . . . . . . . . . . . . . . . . . . . . . . . . . . . . . 500

OKOKOK

14.1.2 Semimartingales. . . . . . . . . . . . . . . . . . . . . . . . . . . . . . . . . . 504

OK

14.1.3 Integrals with respect to jump measures . . . . . . . . . . . . . 507

OK

14.1.4 Levy-type stochastic integrals . . . . . . . . . . . . . . . . . . . . . . 511

OK

14.2 Stochastic differentials . . . . . . . . . . . . . . . . . . . . . . . . . . . . . . . . . . 514

OKOKOK

14.2.1 Ito formula for discontinuous functions . . . . . . . . . . . . . . 514

OKOK

14.2.2 Quadratic variation . . . . . . . . . . . . . . . . . . . . . . . . . . . . . . . 515

OK

14.2.3 Ito formula for semimartingales . . . . . . . . . . . . . . . . . . . . 518

OKOK

14.2.4 Ito formula for Levy processes . . . . . . . . . . . . . . . . . . . . . 520



14.2.5 SDEs with jumps and Ito formula . . . . . . . . . . . . . . . . . . 525

OK

14.2.6 PIDEs and Feynman-Kac representation . . . . . . . . . . . . 529



14.2.7 Linear SDEs with jumps . . . . . . . . . . . . . . . . . . . . . . . . . . 532

OKOKOK this is probably very important too all of this to memorise.

14.3 Levy models with stochastic volatility . . . . . . . . . . . . . . . . . . . . . 534

OK

14.3.1 Levy-driven models and pricing PIDEs . . . . . . . . . . . . . . 534



14.3.2 Bates model . . . . . . . . . . . . . . . . . . . . . . . . . . . . . . . . . . . . . 537

OK

14.3.3 Barndorff-Nielsen and Shephard model . . . . . . . . . . . . . . 539

OK

15 Fourier methods . . . . . . . . . . . . . . . . . . . . . . . . . . . . . . . . . . . . . . . . . . 541



15.1 Characteristic functions and branch cut . . . . . . . . . . . . . . . . . . . 542



15.2 Integral pricing formulas . . . . . . . . . . . . . . . . . . . . . . . . . . . . . . . . 545

OK

15.2.1 Damping method . . . . . . . . . . . . . . . . . . . . . . . . . . . . . . . . . 546

OKOKOK

15.2.2 Pricing formulas. . . . . . . . . . . . . . . . . . . . . . . . . . . . . . . . . . 547



15.2.3 Implementation . . . . . . . . . . . . . . . . . . . . . . . . . . . . . . . . . . 551



15.2.4 Choice of the damping parameter . . . . . . . . . . . . . . . . . . 553

OK

15.3 Fourier-cosine series expansions . . . . . . . . . . . . . . . . . . . . . . . . . . 562

OKOKOK

15.3.1 Implementation . . . . . . . . . . . . . . . . . . . . . . . . . . . . . . . . . . 567



16 Elements of Malliavin calculus . . . . . . . . . . . . . . . . . . . . . . . . . . . . 577

OKOKOK

16.1 Stochastic derivative . . . . . . . . . . . . . . . . . . . . . . . . . . . . . . . . . . . . 578

OKOK

16.1.1 Examples . . . . . . . . . . . . . . . . . . . . . . . . . . . . . . . . . . . . . . . 580



16.1.2 Chain rule . . . . . . . . . . . . . . . . . . . . . . . . . . . . . . . . . . . . . . . 582



16.2 Duality . . . . . . . . . . . . . . . . . . . . . . . . . . . . . . . . . . . . . . . . . . . . . . . 586



16.2.1 Clark-Ocone formula . . . . . . . . . . . . . . . . . . . . . . . . . . . . . . 588

OKOK

16.2.2 Integration by parts and computation of the Greeks . . 590



16.2.3 Examples . . . . . . . . . . . . . . . . . . . . . . . . . . . . . . . . . . . . . . . 594

OKOK

Appendix: a primer in probability and parabolic PDEs . . . . . . . 599



A.1 Probability spaces . . . . . . . . . . . . . . . . . . . . . . . . . . . . . . . . . . . . . . 599



A.1.1 Dynkin's theorems . . . . . . . . . . . . . . . . . . . . . . . . . . . . . . . . 601

OKOK

A.1.2 Distributions . . . . . . . . . . . . . . . . . . . . . . . . . . . . . . . . . . . . 605



A.1.3 Random variables . . . . . . . . . . . . . . . . . . . . . . . . . . . . . . . . 608

OKOK

A.1.4 Integration . . . . . . . . . . . . . . . . . . . . . . . . . . . . . . . . . . . . . . 610

OKOK

A.1.5 Mean and variance . . . . . . . . . . . . . . . . . . . . . . . . . . . . . . . 612



Contents XIII



A.1.6 σ-algebras and information . . . . . . . . . . . . . . . . . . . . . . . . 618

OKOK

A.1.7 Independence . . . . . . . . . . . . . . . . . . . . . . . . . . . . . . . . . . . . 619



A.1.8 Product measure and joint distribution. . . . . . . . . . . . . . 622

OKOK

A.1.9 Markov inequality . . . . . . . . . . . . . . . . . . . . . . . . . . . . . . . . 625



A.2 Fourier transform. . . . . . . . . . . . . . . . . . . . . . . . . . . . . . . . . . . . . . . 626



A.3 Parabolic equations with constant coefficients . . . . . . . . . . . . . . 630



A.3.1 A special case . . . . . . . . . . . . . . . . . . . . . . . . . . . . . . . . . . . . 631

OKOK

A.3.2 General case . . . . . . . . . . . . . . . . . . . . . . . . . . . . . . . . . . . . . 636



A.3.3 Locally integrable initial datum . . . . . . . . . . . . . . . . . . . . 637



A.3.4 Non-homogeneous Cauchy problem . . . . . . . . . . . . . . . . . 638



A.3.5 Adjoint operator . . . . . . . . . . . . . . . . . . . . . . . . . . . . . . . . . 639

OKOKOK

A.4 Characteristic function and normal distribution . . . . . . . . . . . . 641



A.4.1 Multi-normal distribution . . . . . . . . . . . . . . . . . . . . . . . . . 643



A.5 Conditional expectation . . . . . . . . . . . . . . . . . . . . . . . . . . . . . . . . . 646



A.5.1 Radon-Nikodym theorem . . . . . . . . . . . . . . . . . . . . . . . . . . 646

OKOK

A.5.2 Conditional expectation . . . . . . . . . . . . . . . . . . . . . . . . . . . 648

OKOK

A.5.3 Conditional expectation and discrete random variables 650



A.5.4 Properties of the conditional expectation . . . . . . . . . . . . 652



A.5.5 Conditional expectation in L2 . . . . . . . . . . . . . . . . . . . . . . 655

OKOKOK

A.5.6 Change of measure . . . . . . . . . . . . . . . . . . . . . . . . . . . . . . . 656



A.6 Stochastic processes in discrete time . . . . . . . . . . . . . . . . . . . . . . 657

Looks sharp.

A.6.1 Doob's decomposition . . . . . . . . . . . . . . . . . . . . . . . . . . . . . 659



A.6.2 Stopping times . . . . . . . . . . . . . . . . . . . . . . . . . . . . . . . . . . . 661



A.6.3 Doob's maximal inequality . . . . . . . . . . . . . . . . . . . . . . . . 665



A.7 Convergence of random variables . . . . . . . . . . . . . . . . . . . . . . . . . 669



A.7.1 Characteristic function and convergence of variables . . 670



A.7.2 Uniform integrability . . . . . . . . . . . . . . . . . . . . . . . . . . . . . 674



A.8 Topologies and σ-algebras . . . . . . . . . . . . . . . . . . . . . . . . . . . . . . . 676



A.9 Generalized derivatives . . . . . . . . . . . . . . . . . . . . . . . . . . . . . . . . . . 678



A.9.1 Weak derivatives in R . . . . . . . . . . . . . . . . . . . . . . . . . . . . . 678



A.9.2 Sobolev spaces and embedding theorems . . . . . . . . . . . . 681



A.9.3 Distributions . . . . . . . . . . . . . . . . . . . . . . . . . . . . . . . . . . . . 682



A.9.4 Mollifiers . . . . . . . . . . . . . . . . . . . . . . . . . . . . . . . . . . . . . . . . 687



A.10 Separation of convex sets . . . . . . . . . . . . . . . . . . . . . . . . . . . . . . . . 690



References . . . . . . . . . . . . . . . . . . . . . . . . . . . . . . . . . . . . . . . . . . . . . . . . . . . . 691

Perhaps I ought to read through more of these even.

Index . . . . . . . . . . . . . . . . . . . . . . . . . . . . . . . . . . . . . . . . . . . . . . . . . . . . . . . . . 713
Contents
0 Goals of this Book and Global Overview 1
0.1 What is this book? 1
0.2 Why has this book been written? 2
0.3 For whom is this book intended? 2
0.4 Why should I read this book? 2
0.5 The structure of this book 3
0.6 What this book does not cover 4
0.7 Contact, feedback and more information 4
PART I THE CONTINUOUS THEORY OF PARTIAL
DIFFERENTIAL EQUATIONS 5
1 An Introduction to Ordinary Differential Equations 7
1.1 Introduction and objectives 7
1.2 Two-point boundary value problem 8
1.2.1 Special kinds of boundary condition 8
1.3 Linear boundary value problems 9
1.4 Initial value problems 10
1.5 Some special cases 10
1.6 Summary and conclusions 11
2 An Introduction to Partial Differential Equations 13
2.1 Introduction and objectives 13
2.2 Partial differential equations 13
2.3 Specialisations 15
2.3.1 Elliptic equations 15
2.3.2 Free boundary value problems 17
2.4 Parabolic partial differential equations 18
2.4.1 Special cases 20
2.5 Hyperbolic equations 20
2.5.1 Second-order equations 20
2.5.2 First-order equations 21
2.6 Systems of equations 22
2.6.1 Parabolic systems 22
2.6.2 First-order hyperbolic systems 22
2.7 Equations containing integrals 23
2.8 Summary and conclusions 24
3 Second-Order Parabolic Differential Equations 25
3.1 Introduction and objectives 25
3.2 Linear parabolic equations 25
3.3 The continuous problem 26
3.4 The maximum principle for parabolic equations 28
3.5 A special case: one-factor generalised Black-Scholes models 29
3.6 Fundamental solution and the Green's function 30
3.7 Integral representation of the solution of parabolic PDEs 31
3.8 Parabolic equations in one space dimension 33
3.9 Summary and conclusions 35
4 An Introduction to the Heat Equation in One Dimension 37
4.1 Introduction and objectives 37
4.2 Motivation and background 38
4.3 The heat equation and financial engineering 39
4.4 The separation of variables technique 40
4.4.1 Heat flow in a road with ends held at constant temperature 42
4.4.2 Heat flow in a rod whose ends are at a specified
variable temperature 42
4.4.3 Heat flow in an infinite rod 43
4.4.4 Eigenfunction expansions 43
4.5 Transformation techniques for the heat equation 44
4.5.1 Laplace transform 45
4.5.2 Fourier transform for the heat equation 45
4.6 Summary and conclusions 46
5 An Introduction to the Method of Characteristics 47
5.1 Introduction and objectives 47
5.2 First-order hyperbolic equations 47
5.2.1 An example 48
5.3 Second-order hyperbolic equations 50
5.3.1 Numerical integration along the characteristic lines 50
5.4 Applications to financial engineering 53
5.4.1 Generalisations 55
5.5 Systems of equations 55
5.5.1 An example 57
5.6 Propagation of discontinuities 57
5.6.1 Other problems 58
5.7 Summary and conclusions 59
PART II FINITE DIFFERENCE METHODS: THE FUNDAMENTALS 61
6 An Introduction to the Finite Difference Method 63
6.1 Introduction and objectives 63
6.2 Fundamentals of numerical differentiation 63
6.3 Caveat: accuracy and round-off errors 65
6.4 Where are divided differences used in instrument pricing? 67
6.5 Initial value problems 67
6.5.1 Pade matrix approximations 68
6.5.2 Extrapolation 71
6.6 Nonlinear initial value problems 72
6.6.1 Predictor-corrector methods 73
6.6.2 Runge-Kutta methods 74
6.7 Scalar initial value problems 75
6.7.1 Exponentially fitted schemes 76
6.8 Summary and conclusions 76
7 An Introduction to the Method of Lines 79
7.1 Introduction and objectives 79
7.2 Classifying semi-discretisation methods 79
7.3 Semi-discretisation in space using FDM 80
7.3.1 A test case 80
7.3.2 Toeplitz matrices 82
7.3.3 Semi-discretisation for convection-diffusion problems 82
7.3.4 Essentially positive matrices 84
7.4 Numerical approximation of first-order systems 85
7.4.1 Fully discrete schemes 86
7.4.2 Semi-linear problems 87
7.5 Summary and conclusions 89
8 General Theory of the Finite Difference Method 91
8.1 Introduction and objectives 91
8.2 Some fundamental concepts 91
8.2.1 Consistency 93
8.2.2 Stability 93
8.2.3 Convergence 94
8.3 Stability and the Fourier transform 94
8.4 The discrete Fourier transform 96
8.4.1 Some other examples 98
8.5 Stability for initial boundary value problems 99
8.5.1 Gerschgorin's circle theorem 100
8.6 Summary and conclusions 101
9 Finite Difference Schemes for First-Order Partial Differential Equations 103
9.1 Introduction and objectives 103
9.2 Scoping the problem 103
9.3 Why first-order equations are different: Essential difficulties 105
9.3.1 Discontinuous initial conditions 106
9.4 A simple explicit scheme 106
9.5 Some common schemes for initial value problems 108
9.5.1 Some other schemes 110
9.6 Some common schemes for initial boundary value problems 110
9.7 Monotone and positive-type schemes 110
9.8 Extensions, generalisations and other applications 111
9.8.1 General linear problems 112
9.8.2 Systems of equations 112
9.8.3 Nonlinear problems 114
9.8.4 Several independent variables 114
9.9 Summary and conclusions 115
10 FDM for the One-Dimensional Convection-Diffusion Equation 117
10.1 Introduction and objectives 117
10.2 Approximation of derivatives on the boundaries 118
10.3 Time-dependent convection-diffusion equations 120
10.4 Fully discrete schemes 120
10.5 Specifying initial and boundary conditions 121
10.6 Semi-discretisation in space 121
10.7 Semi-discretisation in time 122
10.8 Summary and conclusions 122
11 Exponentially Fitted Finite Difference Schemes 123
11.1 Introduction and objectives 123
11.2 Motivating exponential fitting 123
11.2.1 'Continuous' exponential approximation 124
11.2.2 'Discrete' exponential approximation 125
11.2.3 Where is exponential fitting being used? 128
11.3 Exponential fitting and time-dependent convection-diffusion 128
11.4 Stability and convergence analysis 129
11.5 Approximating the derivative of the solution 131
11.6 Special limiting cases 132
11.7 Summary and conclusions 132
PART III APPLYING FDM TO ONE-FACTOR INSTRUMENT PRICING 135
12 Exact Solutions and Explicit Finite Difference Method
for One-Factor Models 137
12.1 Introduction and objectives 137
12.2 Exact solutions and benchmark cases 137
12.3 Perturbation analysis and risk engines 139
12.4 The trinomial method: Preview 139
12.4.1 Stability of the trinomial method 141
12.5 Using exponential fitting with explicit time marching 142
12.6 Approximating the Greeks 142
12.7 Summary and conclusions 144
12.8 Appendix: the formula for Vega 144
13 An Introduction to the Trinomial Method 147
13.1 Introduction and objectives 147
13.2 Motivating the trinomial method 147
13.3 Trinomial method: Comparisons with other methods 149
13.3.1 A general formulation 150
13.4 The trinomial method for barrier options 151
13.5 Summary and conclusions 152
14 Exponentially Fitted Difference Schemes for Barrier Options 153
14.1 Introduction and objectives 153
14.2 What are barrier options? 153
14.3 Initial boundary value problems for barrier options 154
14.4 Using exponential fitting for barrier options 154
14.4.1 Double barrier call options 156
14.4.2 Single barrier call options 156
14.5 Time-dependent volatility 156
14.6 Some other kinds of exotic options 157
14.6.1 Plain vanilla power call options 158
14.6.2 Capped power call options 158
14.7 Comparisons with exact solutions 159
14.8 Other schemes and approximations 162
14.9 Extensions to the model 162
14.10 Summary and conclusions 163
15 Advanced Issues in Barrier and Lookback Option Modelling 165
15.1 Introduction and objectives 165
15.2 Kinds of boundaries and boundary conditions 165
15.3 Discrete and continuous monitoring 168
15.3.1 What is discrete monitoring? 168
15.3.2 Finite difference schemes and jumps in time 169
15.3.3 Lookback options and jumps 170
15.4 Continuity corrections for discrete barrier options 171
15.5 Complex barrier options 171
15.6 Summary and conclusions 173
16 The Meshless (Meshfree) Method in Financial Engineering 175
16.1 Introduction and objectives 175
16.2 Motivating the meshless method 175
16.3 An introduction to radial basis functions 177
16.4 Semi-discretisations and convection-diffusion equations 177
16.5 Applications of the one-factor Black-Scholes equation 179
16.6 Advantages and disadvantages of meshless 180
16.7 Summary and conclusions 181
17 Extending the Black-Scholes Model: Jump Processes 183
17.1 Introduction and objectives 183
17.2 Jump-diffusion processes 183
17.2.1 Convolution transformations 185
17.3 Partial integro-differential equations and financial applications 186
17.4 Numerical solution of PIDE: Preliminaries 187
17.5 Techniques for the numerical solution of PIDEs 188
17.6 Implicit and explicit methods 188
17.7 Implicit-explicit Runge-Kutta methods 189
17.8 Using operator splitting 189
17.9 Splitting and predictor-corrector methods 190
17.10 Summary and conclusions 191
PART IV FDM FOR MULTIDIMENSIONAL PROBLEMS 193
18 Finite Difference Schemes for Multidimensional Problems 195
18.1 Introduction and objectives 195
18.2 Elliptic equations 195
18.2.1 A self-adjoint elliptic operator 198
18.2.2 Solving the matrix systems 199
18.2.3 Exact solutions to elliptic problems 200
18.3 Diffusion and heat equations 202
18.3.1 Exact solutions to the heat equation 204
18.4 Advection equation in two dimensions 205
18.4.1 Initial boundary value problems 207
18.5 Convection-diffusion equation 207
18.6 Summary and conclusions 208
19 An Introduction to Alternating Direction Implicit and Splitting Methods 209
19.1 Introduction and objectives 209
19.2 What is ADI, really? 210
19.3 Improvements on the basic ADI scheme 212
19.3.1 The D'Yakonov scheme 212
19.3.2 Approximate factorization of operators 213
19.3.3 ADI classico for two-factor models 215
19.4 ADI for first-order hyperbolic equations 215
19.5 ADI classico and three-dimensional problems 217
19.6 The Hopscotch method 218
19.7 Boundary conditions 219
19.8 Summary and conclusions 221
20 Advanced Operator Splitting Methods: Fractional Steps 223
20.1 Introduction and objectives 223
20.2 Initial examples 223
20.3 Problems with mixed derivatives 224
20.4 Predictor-corrector methods (approximation correctors) 226
20.5 Partial integro-differential equations 227
20.6 More general results 228
20.7 Summary and conclusions 228
21 Modern Splitting Methods 229
21.1 Introduction and objectives 229
21.2 Systems of equations 229
21.2.1 ADI and splitting for parabolic systems 230
21.2.2 Compound and chooser options 231
21.2.3 Leveraged knock-in options 232
21.3 A different kind of splitting: The IMEX schemes 232
21.4 Applicability of IMEX schemes to Asian option pricing 234
21.5 Summary and conclusions 235
PART V APPLYING FDM TO MULTI-FACTOR INSTRUMENT PRICING 237
22 Options with Stochastic Volatility: The Heston Model 239
22.1 Introduction and objectives 239
22.2 An introduction to Ornstein-Uhlenbeck processes 239
22.3 Stochastic differential equations and the Heston model 240
22.4 Boundary conditions 241
22.4.1 Standard european call option 242
22.4.2 European put options 242
22.4.3 Other kinds of boundary conditions 242
22.5 Using finite difference schemes: Prologue 243
22.6 A detailed example 243
22.7 Summary and conclusions 246
23 Finite Difference Methods for Asian Options and Other 'Mixed' Problems 249
23.1 Introduction and objectives 249
23.2 An introduction to Asian options 249
23.3 My first PDE formulation 250
23.4 Using operator splitting methods 251
23.4.1 For sake of completeness: ADI methods for asian option PDEs 253
23.5 Cheyette interest models 253
23.6 New developments 254
23.7 Summary and conclusions 255
24 Multi-Asset Options 257
24.1 Introduction and objectives 257
24.2 A taxonomy of multi-asset options 257
24.2.1 Exchange options 260
24.2.2 Rainbow options 261
24.2.3 Basket options 262
24.2.4 The best and worst 263
24.2.5 Quotient options 263
24.2.6 Foreign equity options 264
24.2.7 Quanto options 264
24.2.8 Spread options 264
24.2.9 Dual-strike options 265
24.2.10 Out-perfomance options 265
24.3 Common framework for multi-asset options 265
24.4 An overview of finite difference schemes for multi-asset problems 266
24.5 Numerical solution of elliptic equations 267
24.6 Solving multi-asset Black-Scholes equations 269
24.7 Special guidelines and caveats 270
24.8 Summary and conclusions 271
25 Finite Difference Methods for Fixed-Income Problems 273
25.1 Introduction and objectives 273
25.2 An introduction to interest rate modelling 273
25.3 Single-factor models 274
25.4 Some specific stochastic models 276
25.4.1 The Merton model 277
25.4.2 The Vasicek model 277
25.4.3 Cox, Ingersoll and Ross (CIR) 277
25.4.4 The Hull-White model 277
25.4.5 Lognormal models 278
25.5 An introduction to multidimensional models 278
25.6 The thorny issue of boundary conditions 280
25.6.1 One-factor models 280
25.6.2 Multi-factor models 281
25.7 Introduction to approximate methods for interest rate models 282
25.7.1 One-factor models 282
25.7.2 Many-factor models 283
25.8 Summary and conclusions 283
PART VI FREE AND MOVING BOUNDARY VALUE PROBLEMS 285
26 Background to Free and Moving Boundary Value Problems 287
26.1 Introduction and objectives 287
26.2 Notation and definitions 287
26.3 Some preliminary examples 288
26.3.1 Single-phase melting ice 288
26.3.2 One-factor option modelling: American exercise style 289
26.3.3 Two-phase melting ice 290
26.3.4 The inverse Stefan problem 290
26.3.5 Two and three space dimensions 291
26.3.6 Oxygen diffusion 293
26.4 Solutions in financial engineering: A preview 293
26.4.1 What kinds of early exercise features? 293
26.4.2 What kinds of numerical techniques? 294
26.5 Summary and conclusions 294
27 Numerical Methods for Free Boundary Value Problems:
Front-Fixing Methods 295
27.1 Introduction and objectives 295
27.2 An introduction to front-fixing methods 295
27.3 A crash course on partial derivatives 295
27.4 Functions and implicit forms 297
27.5 Front fixing for the heat equation 299
27.6 Front fixing for general problems 300
27.7 Multidimensional problems 300
27.8 Front fixing and American options 303
27.9 Other finite difference schemes 305
27.9.1 The method of lines and predictor-corrector 305
27.10 Summary and conclusions 306
28 Viscosity Solutions and Penalty Methods for American Option Problems 307
28.1 Introduction and objectives 307
28.2 Definitions and main results for parabolic problems 307
28.2.1 Semi-continuity 307
28.2.2 Viscosity solutions of nonlinear parabolic problems 308
28.3 An introduction to semi-linear equations and penalty method 310
28.4 Implicit, explicit and semi-implicit schemes 311
28.5 Multi-asset American options 312
28.6 Summary and conclusions 314
29 Variational Formulation of American Option Problems 315
29.1 Introduction and objectives 315
29.2 A short history of variational inequalities 316
29.3 A first parabolic variational inequality 316
29.4 Functional analysis background 318
29.5 Kinds of variational inequalities 319
29.5.1 Diffusion with semi-permeable membrane 319
29.5.2 A one-dimensional finite element approximation 320
29.6 Variational inequalities using Rothe's methods 323
29.7 American options and variational inequalities 324
29.8 Summary and conclusions 324
PART VII DESIGN AND IMPLEMENTATION IN C++ 325
30 Finding the Appropriate Finite Difference Schemes for your Financial
Engineering Problem 327
30.1 Introduction and objectives 327
30.2 The financial model 328
30.3 The viewpoints in the continuous model 328
30.3.1 Payoff functions 329
30.3.2 Boundary conditions 330
30.3.3 Transformations 331
30.4 The viewpoints in the discrete model 332
30.4.1 Functional and non-functional requirements 332
30.4.2 Approximating the spatial derivatives in the PDE 333
30.4.3 Time discretisation in the PDE 334
30.4.4 Payoff functions 334
30.4.5 Boundary conditions 335
30.5 Auxiliary numerical methods 335
30.6 New Developments 336
30.7 Summary and conclusions 336
31 Design and Implementation of First-Order Problems 337
31.1 Introduction and objectives 337
31.2 Software requirements 337
31.3 Modular decomposition 338
31.4 Useful C++ data structures 339
31.5 One-factor models 339
31.5.1 Main program and output 342
31.6 Multi-factor models 343
31.7 Generalisations and applications to quantitative finance 346
31.8 Summary and conclusions 347
31.9 Appendix: Useful data structures in C++ 348
32 Moving to Black-Scholes 353
32.1 Introduction and objectives 353
32.2 The PDE model 354
32.3 The FDM model 355
32.4 Algorithms and data structures 355
32.5 The C++ model 356
32.6 Test case: The two-dimensional heat equation 357
32.7 Finite difference solution 357
32.8 Moving to software and method implementation 358
32.8.1 Defining the continuous problem 358
32.8.2 Creating a mesh 358
32.8.3 Choosing a scheme 360
32.8.4 Termination criterion 361
32.9 Generalisations 361
32.9.1 More general PDEs 361
32.9.2 Other finite difference schemes 361
32.9.3 Flexible software solutions 361
32.10 Summary and conclusions 362
33 C++ Class Hierarchies for One-Factor and Two-Factor Payoffs 363
33.1 Introduction and objectives 363
33.2 Abstract and concrete payoff classes 364
33.3 Using payoff classes 367
33.4 Lightweight payoff classes 368
33.5 Super-lightweight payoff functions 369
33.6 Payoff functions for multi-asset option problems 371
33.7 Caveat: non-smooth payoff and convergence degradation 373
33.8 Summary and conclusions 374
Appendices 375
A1 An introduction to integral and partial integro-differential equations 375
A2 An introduction to the finite element method 393
Bibliography 409
Index 417
Greene
Wooldridge
Gujarati
Long
Hayashi
Hansen
Econometrica
Chris
Cameron
Trivedi
Angrist And Pischke
DiNardo
Kennedy
Davidson And MacKinnon
Taylor
Tsay
Johnston
Lambert
Verbeek
Porter
Montiel
...

I was considering trying to find the textbooks directly corresponding with Pennsylvania or the University Of Texas At Austin.

I could copy and paste the source comments here:

"
Econometrics covers a vast area, time series covers a big part of econometrics, we also have panel data and cross-section data to deal with .

Econometric Analysis, Willam H. Greene, Stern School of Business, New York University - excellent book for under graduates. It covers basics of econometrics - statistical inference/estimation, introduces cross-sectional data and panel data concepts.
Time Series is usually introduced to students in later stage in second year or start of third year. There are some good books by Granger (everyone knew him liked him) and Newbold. Their books are more of applied (no maths proves), but truly excellent introduction to this topic. Unit root topic is covered extensively. Granger was the pioneer in discovery of co-integrations.
The Econometrics of Financial Markets Hardcover – 29 Dec. 1996 ,by John Y. Campbell (Author), Andrew W. Lo (Author), A. Craig MacKinlay (Author). An excellent book truly.
Time Series Analysis by James D. Hamilton, excellent book for mathematical proves, introduces Kalman filter concept. It is probably not for undergrad.
If you are just starting, look up for good regression analysis books and introduction to statistical estimation/inference.
"

"
I sometimes get asked what is a “good” book for learning econometrics or statistics. To avoid me giving an incomplete or ill thought-out answer, I list a few of my favourites here,

“Mastering Metrics” by Josh Angrist and Jörn-Steffen Pischke. This is the best introductory text on causal inference that exists. Its chapters guide the student through the five main paths to causal inference, including: randomisation, regression, instrumental variables, regression discontinuity designs and differences-in-differences. The text is written in a style that means it has much to offer researchers whilst still being accessible to aspiring inferers.
“Mostly Harmless Econometrics” (again) by Josh Angrist and Jörn-Steffen Pischke. This is the best non-introductory text on causal inference that exists. Don’t let the slim size of this book fool you. This isn’t a Jo Nesbo novel. Behind its meagre paper weight, it comprises seriously weighty content. It covers the predominant paths to causal inference, including all those topics discussed in Mastering Metrics, and more, like propensity score matching. Owning this book is essential for all serious practitioners of inference from observational data.
“Bayesian Data Analysis” by Andrew Gelman et al. (be sure to get the third edition). This is the most comprehensive text on Bayesian analysis that exists. Whilst cover-to-cover reading is perhaps not encouraged, I find myself dipping into individual chapters time and time again. Note: this book is quite mathematically advanced. This is part of the reason I wrote my book (see below).
“Introductory Econometrics” by Jeffrey Wooldridge. This is the book that ignited my interest in econometrics. It is written in a very accessible way and – whilst I would argue is a little bit dated now – is probably the best introductory text on classical econometrics.
“A Student’s Guide to Bayesian Statistics” by me (sorry for this somewhat shameless plug). Hopefully does what is says. The text does not assume any previous knowledge of statistics – classical or Bayesian – nor probability. The content is about as un-mathematical as I could make it without turning it into a novel (about statistics). The book, instead of focussing on the maths, highlights the intuition behind Bayesian inference and MCMC algorithms, working up to a practical introduction to Stan and hierarchical models.
"

"
econometricsbooks.com
 

Books
On-line Books / Notes
 
This website provides information on econometrics books and on-line resources.

Econometrics books
Undergraduate textbooks
Graduate textbooks
Other topics:  Bayesian econometrics, bootstrap, business statistics/econometrics, count data, duration data, macroeconometrics, mathematics, Monte Carlo methods, nonparametric econometrics, panel data, quantile regression, network data, Stata, theoretical probability, theoretical statistics, time series
Undergraduate textbooks
Introductory Econometrics WooldridgeIntroductory Econometrics: A Modern Approach 7th edition by Jeffrey M. Wooldridge, South-Western College Publishers (2018, 816pp) --- This textbook contains a comprehensive treatment of undergraduate econometrics.  The book is a bit more technical than some other undergraduate texts (like Stock and Watson's "Introduction to Econometrics"), but most of the advanced mathematics is left as optional material.  Wooldridge explains concepts very clearly.  The topic coverage goes well beyond the multiple regression model, hitting important topics like binary-choice models, panel-data models, and forecasting techniques.  Examples are abundant throughout the text, and many end-of-chapter exercises (based on real datasets) provide students the opportunity to fully absorb the book's material.

Using R for Introductory EconometricsUsing R for Introductory Econometrics 2nd edition by Florian Heiss (2020, 378pp) --- This book provides an introduction to R by providing code and scripts that follow closely to the examples in Wooldridge's Introductory Econometrics textbook (6th Edition).  This book provides a great way to give students a background in R, even if they do so on a self-study basis.

Using Python for Introductory EconometricsUsing Python for Introductory Econometrics 1st edition by Florian Heiss (2020, 428pp) --- This book provides an introduction to Python by providing code that follows closely to the examples in Wooldridge's Introductory Econometrics textbook (6th Edition).  This book provides a great way to give students a background in Python, even if they do so on a self-study basis.

Introduction to Econometrics Stock WatsonIntroduction to Econometrics 4th edition by James H. Stock and Mark W. Watson, Addison-Wesley (2018, 800pp) --- This textbook by Stock and Watson is an excellent option for a student's first exposure to econometrics.  The book is extremely well written, with interesting empirical examples used to motivate the importance of econometrics to the field of economics.  The book is primarily targeted for a less theoretical course in econometrics, but there are several chapters that could be used in a more formal course.  Additional resources:  book website.

Causal Inference The Mixtape CunninghamCausal Inference: The Mixtape by Scott Cunningham, Yale University Press (2021, 512pp) --- This book by Cunningham is the newest entry among causal inference books for economists. It's a great alternative to Mostly Harmless Econometrics with very interesting applications (datasets and programs (Stata and R) all included).   Additional resources:  book website.

Mastering Metrics Angrist PischkeMastering 'Metrics: The Path from Cause to Effect by Joshua D. Angrist and Jorn-Steffen Pischke, Princeton University Press (2014, 304pp) --- This book by Angrist and Pischke is a gentler introduction to causal inference than their "Mostly Harmless Econometrics" (see below). The book can be used as a companion to Wooldridge or Stock-Watson in an undergraduate course or by itself as a standalone textbook in a second econometrics course.   Additional resources:  book website.

Essentials of Econometrics 4th edition by Damodar Gujarati and Dawn Porter, McGraw-Hill/Irwin (2009, 576pp) --- Book website, author profile

Principles of Econometrics 4th edition by R. Carter Hill, William E. Griffiths, and Guay C. Lim, Wiley (2011, 758pp) --- book website, datasets, a free ebook (by Prof. Lee Adkins) for this textbook using the gretl software package, an EViews guide for Principles of Econometrics, a Stata guide for Principles of Econometrics

Introduction to Econometrics 4th edition by Christopher Dougherty, Oxford University Press (2011, 512pp) --- book website, datasets, lecture slides (from Prof. Dougherty), Google preview, free download of gretl software

Using Econometrics: A Practical Guide 6th edition by A. H. Studenmund, Addison-Wesley (2010, 648pp) --- book website, datasets

Econometric Models and Economic Forecasts by Robert S. Pindyck and Daniel L. Rubinfeld, McGraw-Hill (1997, 634pp) --- Publisher website.  Sample programs for textbook exercises.

Statistics and Econometrics: Methods and Applications by Orley Ashenfelter, Phillip B. Levine, and David J. Zimmerman, Wiley (2006, 320pp) --- Datasets.  Lecture slides.

Econometrics by Example by Damodar Gujarati, Palgrave Macmillan (2014, 496pp) --- book website, datasets (Stata format), Google preview

Graduate textbooks
Econometric Analysis of Cross Section and Panel Data WooldridgeEconometric Analysis of Cross Section and Panel Data 2nd edition by Jeffrey M. Wooldridge, MIT Press (2010, 1096pp) --- Unlike many graduate econometrics textbooks, this book jumps right into the asymptotic treatment of linear regression models (rather than spending the first few chapters on finite-sample results).  The topic coverage is impressive.  The first part of the book devoted to estimation of linear regression models (single equation, multiple equation, with and without endogeneity).  After a couple of chapters on the theory of different estimators (M-estimators, maximum likelihood estimators), the book proceeds to cover a wide range of models, including panel data models, binary choice models, censored/selection models, count data models, etc.  Additional resources:  book website, datasets, solutions (odd-numbered problems), errata, Stata examples from the textbook, sample chapter, Google preview.

Econometric AnalysisEconometric Analysis 8th edition by William H. Greene, Prentice Hall (2017, 1176pp) --- Greene's Econometric Analysis, now in its 8th edition, has long stood as a prominent choice among first-year graduate econometrics textbooks.  The book covers an amazing number of different topics, ranging from finite sample to asymptotic, cross-sectional to time-series, frequentist to Bayesian, and so on.  The appendices themselves are extremely valuable, serving as review material on linear algebra and statistical theory (without having to reference other sources for these topics).  Additional resources:  book website, datasets, solutions manual.

Econometrics HayashiEconometrics by Fumio Hayashi, Princeton University Press (2000, 690pp) --- Hayashi's Econometrics textbook provides a modern formal treatment of graduate econometrics.  The book uses generalized method of moments (GMM) estimation as a unifying tool for many of the topics that are covered.  (Maximum likelihood methods, and the connected models, are considered separately.)  The book starts with linear regression and continues on to models with endogeneity and panel-data models.  Unlike some graduate textbooks, this book has a serious treatment of both stationary and nonstationary time-series models.  An impressive amount of material is covered both concisely and clearly.  Additional resources:  book website, datasets, some solutions, sample chapter, errata, revised chapter on maximum likelihood.

Microeconometrics Methods and Applications Cameron and TrivediMicroeconometrics: Methods and Applications by A. Colin Cameron and Pravin K. Trivedi, Cambridge University Press (2005, 1056pp) --- The authors cover an impressive amount of material in this textbook, covering nearly every model used in empirical microeconomics today.  Illustrative examples and datasets are used throughout.  Additional resources:  Book website, datasets, some solutions and extra exercises, errata, Google preview.  The book Microeconometrics Using Stata (2009) by Cameron and Trivedi is a valuable resource for Stata users, covering much of the same materials and empirical examples as the Microeconometrics: Methods and Applications textbook.

Mostly Harmless EconometricsMostly Harmless Econometrics: An Empiricist's Companion by Joshua D. Angrist and Jorn-Steffen Pischke, Princeton University Press (2008, 392pp) --- This book is not your standard econometrics textbook but is already popular among empirical economists.  Rather than focusing too much on theory and models, the authors use examples from published papers to illustrate some of the most common econometric techniques available to today's researchers.  The book would make a nice companion to a more theoretical textbook.  Additional resources:  book website, datasets, sample chapter from publisher, Google preview.

Learning Microeconometrics with R by Christopher P. Adams, Chapman and Hall/CRC (2020, 398pp) --- This book provides an introduction to the field of microeconometrics through the use of R. The focus is on applying current learning from the field to real world problems. It uses R to both teach the concepts of the field and show the reader how the techniques can be used.   Additional resources:  book website and resources.

Matching, Regression Discontinuity, Difference in Differences, and Beyond by Myoung-jae Lee, Oxford University Press (2016, 280pp) --- This book reviews the three most popular methods (and their extensions) in applied economics and other social sciences: matching, regression discontinuity, and difference in differences.   It book introduces the underlying econometric and statistical ideas, shows what is identified and how the identified parameters are estimated, and illustrates how they are applied with real empirical examples.   Additional resources:  publisher website, Google preview.

A Guide to EconometricsA Guide to Econometrics 6th edition by Peter Kennedy, Wiley-Blackwell (2008, 600pp) --- This book is Kennedy's sixth edition of the "Guide", which has been a widely used reference since its early editions.  The "Guide" is not a replacement for your favorite econometrics textbook, but it is a useful reference to have on your bookshelf.  A large part of the book (roughly 200 pages) is devoted to the classical regression model and the various violations of its assumptions.  The book then hits the big ideas within several additional topics, including limited-dependent variable models, panel-data models, and time-series models.  Additional resources:  book website, datasets, book preface.

A Course in Econometrics by Arthur S. Goldberger, Harvard University Press (1991, 432pp) --- A shorter option than most modern graduate textbooks, but it provides a concise and rigorous treatment of the material.  Google preview.  Publisher website.

An Introduction to Classical Econometric Theory by Paul A. Ruud, Oxford University Press (2000, 976pp) --- Prof. Ruud provides some of the most rigorous treatments of topics within first-year graduate econometrics.  His use of projections and graphical intuition for regression is very appealing.  Author's textbook website (with solutions, errata, and datasets).

Bayesian econometrics
Introduction to Modern Bayesian Econometrics by Tony Lancaster, Wiley-Blackwell (2004, 416pp) --- Google preview.

Contemporary Bayesian Econometrics and Statistics by John Geweke, Wiley-Interscience (2005, 320pp) --- Google preview.

Bayesian Econometrics by Gary Koop, Wiley-Interscience (2003, 376pp) --- Google preview.

Bayesian Econometric Methods (Econometric Exercises) by Gary Koop, Dale J. Poirier, & Justin L. Tobias, Cambridge University Press (2019, 480pp) ---  This book provides an impressive collection of exercises and solutions for Bayesian econometrics.  Google preview.

Complete and Incomplete Econometrics Models by John Geweke (2010, 176pp) --- Google preview.

Introduction to Bayesian Econometrics by Edward Greenberg, Cambridge University Press (2007, 224pp) --- resource webpage, Google preview.

Oxford Handbook of Bayesian Econometrics Oxford University Press (2011, 560pp)

Bootstrap
An Introduction to the Bootstrap by Bradley Efron and R. J. Tibshirani, Chapman & Hall (1994, 456pp) --- This book provides an easy-to-read introduction to the bootstrap.

Bootstrap Methods and Their Applications by A. C. Davison and D. V. Hinkley, Cambridge University Press (1997, 594pp) --- This book is another easy-to-read guide to the bootstrap for applied researchers.  Several statistical models, in addition to the linear regression model, are covered.

An Introduction to Bootstrap Methods with Applications to R by Michael R. Chernick and Robert A. LaBudde, Wiley (2011, 240pp) --- This relatively short book provides a comprehensive guide to implementation of the bootstrap in the open-source statistical platform R.

Business statistics/econometrics
Applied Regression Analysis by Terry E. Dielman, South-western (2004, 496pp)

Statistics for Business and Economics 11th edition by David R. Anderson, Dennis J. Sweeney, and Thomas A. Williams, South-western (2011, 1120pp) --- Book website.

Count data
Regression Analysis of Count Data by A. Colin Cameron & Pravin K. Trivedi, Cambridge University Press (2013, 432pp) --- This overview of count-data models and estimation methods was written prior to Cameron and Trivedi's more general books in graduate econometrics.  The 2nd edition was published in 2013.

Duration data
The Econometric Analysis of Transition Data by Tony Lancaster, Cambridge University Press (2008, 368pp) --- Prof. Lancaster's classic book on duration models is now available in its second edition, updated to account for recent developments in the field.  Google preview (note that this preview is for the first edition).

Macroeconometrics
Structural Macroeconometrics by David DeJong and Chetan Dave, Princeton University Press (2011, 428pp) --- code and datasets, Google preview.

Mathematics
Mathematics for Econometrics by Phoebus J. Dhrymes, Springer (2013, 419pp) --- Google preview.

Monte Carlo methods
Monte Carlo Simulation for Econometricians by Jan Kiviet, NOW Publishers (2012, 198pp) --- access to EViews programs.

Nonparametric econometrics
Nonparametric Econometrics: Theory and Practice by Qi Li & Jeffrey Racine, Princeton University Press (2006, 768pp) --- Preface available. 

Nonparametric Econometrics by Adrian Pagan & Aman Ullah, Cambridge University Press (1999, 444pp) --- Google preview.

Applied Nonparametric Econometrics by Daniel J. Henderson and Christopher F. Parmeter, Cambridge University Press (2015, 378pp) --- book website.

Nonparametric Econometrics: Theory and Practice by Qi Li & Jeffrey Racine, Princeton University Press (2006, 768pp) --- Preface available. 

Panel data
Econometric Analysis of Panel Data by Badi Baltagi, Wiley (2013, 390pp) --- Now in its fifth edition, this book covers a wide range of panel-data models (linear, non-linear, dynamic, etc) and their associated estimators.  A companion volume is also available.

Analysis of Panel Data by Cheng Hsiao, Cambridge University Press (2014, 562pp) --- Prof. Hsiao updated his classic book on panel data in this third edition and now covers many of the developments of the last 30 years. Google preview.

Panel Data Econometrics by Manuel Arellano, Oxford University Press (2003, 248pp) --- Prof. Arellano covers many panel-data topics, including a very complete treatment of dynamic linear panel-data models.  Google preview.

Time Series and Panel Data Econometrics by M. Hashem Pesaran, Oxford University Press (2015, 1062pp) --- This book is concerned with recent developments in time series and panel data techniques for the analysis of macroeconomic and financial data. It provides a rigorous, nevertheless user-friendly, account of the time series techniques dealing with univariate and multivariate time series models, as well as panel data models.

Quantile regression
Quantile Regression by Roger Koenker, Cambridge University Press (2005, 366pp) --- This book, written by the foremost expert on quantile regression, contains a comprehensive treatment of quantile regression models and estimation methods.

Network data
The Econometric Analysis of Network Data edited by Bryan Graham and Aureo de Paula, Academic Press (2005, 320pp) --- This book serves as an entry point for advanced students, researchers, and data scientists seeking to perform effective analyses of networks, especially inference problems. It introduces the key results and ideas in an accessible, yet rigorous way, confining formal proofs to extensively annotated appendices. While a multi-contributor references, the work is tightly focused and disciplined, providing latitude for varied specialties in one authorial voice.

Stata
An Introduction to Modern Econometrics Using Stata by Christopher F. Baum, Stata Press (2006, 340pp) --- This book provides an excellent guide to econometrics using the Stata software package.  Author's webpage for the book (including datasets, programs, and errata).

An Introduction to Stata Programming by Christopher F. Baum, Stata Press (2015, 412pp)

Microeconometrics Using Stata by A. Colin Cameron and Pravin K. Trivedi, Stata Press (2010, 706pp)

Introduction to Time Series Using Stata by Sean Becketti, Stata Press (2013, 741pp)

An Introduction to Survival Analysis Using Stata by Cleves, Gould, Gutierrez, and Marchenko, Stata Press (2010, 706pp)

Theoretical probability
A User's Guide to Measure Theoretic Probability by David Pollard, Cambridge University Press (2001, 320pp) --- Google preview.

Probability and Measure by Patrick Billingsley, Wiley (1995, 608pp)

Theoretical statistics
Approximation Theorems of Mathematical Statistics by Robert J. Serfling, Wiley (2001, 400pp)

Asymptotic Statistics by Aad van der Vaart, Cambridge University Press (2000, 460pp) --- Google preview.

Introduction to Empirical Processes and Semiparametric Inference by Michael R. Kosorok, Springer (2008, 483pp) --- Google preview.

Weak Convergence and Empirical Processes: With Applications to Statistics by Aad van der Vaart and Jon Wellner, Springer (2000, 532pp) --- Google preview.

Time Series
Analysis of Financial Time Series by Ruey Tsay, Wiley (2010, 677pp) --- Author's webpage with errata, datasets, and more.  Google preview.

An Introduction to Analysis of Financial Data with R by Ruey Tsay, Wiley (2012, 416pp) --- Author's webpage with errata, datasets, and more.  Google preview.

Time Series Analysis by James Hamilton, Princeton University Press (1994, 820pp) --- Data and software for the textbook.  Google preview.

Applied Econometric Time Series by Walter Enders, Wiley (2014, 496pp) --- Author's textbook webpage.

Introductory Econometrics for Finance by Chris Brooks, Cambridge University Press (2014, 740pp) --- book website.

Econometrics of Financial High-Frequency Data by Nikolas Hautsch, Springer (2011, 386pp)

 

Books | On-line books/notes
"
\Large

\textbf{Introduction}

Bound from monovariant being the sum of $\frac{n_i}{2^i}$ is monovariant under the operation.

1 \\
Below one can construct relevant monovariant(s) and invariant(s).

2 \\
Induction on the number of buckets bound argument will yield the conclusion in conjunction with the observation of the monovariant that is the number of pennies in the leftmost positive bucket e.g. is monotone decreasing or recurse in to the inductive hypothesis sub case between such moves.

3 \\
Induce symmetry in to a square of side length $7$ and obtain $\boxed{7\sqrt{2}}$ or on sight the fact that the powerful hammer of coordinate bashing works and execute.

4 \\
IMO.

5 \\
Well for example a key idea here is the decomposition in to a factorization and a prime factorization and namely one can set $36x+y=2^a,x+36y=2^b$ and obtain $x=\frac{9 \cdot 2^{a+2}-2^b}{1295},y=\frac{9 \cdot 2^{b+2}-2^a}{1295}$ whence a contradiction as positivity of $x,y$ imply parity of numerators is even implies $x,y$ are non integers e.g.

6 \\
$\boxed{\frac{2}{3}}$ factor telescope.

7 \\
Well $\boxed{a_0,a_1,\dots,a_n=1,0,1,0,\dots,1,0}$

8 \\
StackExchange, UTK, for further reading Niven. One can utilize Chebyshev Polynomials or a trigonometry identity based recursion contradiction integer modulo argumentation via $e^{ia}=\cos(a)+i\sin(a)$

9 \\
$\boxed{\text{Yes}}$

\newpage

\textbf{Polynomials}

1 \\
It follows by Vieta's and the fact that if the line was $y=ax+b$ then the intersection of the line and the curve would occur at the zeros of $x^3-(ax+b)$ which then have sum $0$.

2 \\
Intersection at most one more time follows from degree and rationality of this point follows algebraically otherwise there would be a contradiction via Vieta's e.g.

3 \\
It follows algebraically after noting that every term in $P$ must have even degree.

4 \\
Algebra gives $|b|<1,|a|\le 1+b$

5 \\
Seems like some analysis.

6 \\
Maybe there exists some obvious construction.

7 \\
A Problem Seminar.

\newpage

\textbf{Number Theory}

1 \\
$2^{29}=536870912$ so $\boxed{4}$ or note that $2^{29} \equiv 5 \pmod{9}$ and also from the digit sum is $0+1+2+\dots+9-n \equiv 45-n \equiv -n \pmod{9}$

2 \\
Imitating Euclid, suppose not and there are finitely many $a_i$ but then $2a_1a_2\dots a_n+1 \equiv 3 \pmod{4}$ thus it must contain another prime $\equiv 3 \pmod{4}$ in its prime factorization, contradiction.

3 \\
Combinatorially it suffices to count the number of selections of $pk$ elements from $pn$ elements which do not form a cyclic permutation set of selections under rotation by $n$ but those must just be the selections which are already cyclic modulo $n$ and there are precisely then $\binom{n}{k}$ such.

4 \\
$5^N-5^n \equiv 0 \pmod{10^n}$ but indeed there exists such an $N$ as all we need is $5^n \equiv 5^n \pmod{2^n}$ or rather $5^{N-n} \equiv 1 \pmod{2^n}$ but by Euler it suffices to take $N=n+\phi(2^n)=n+2^{n-1}$

5 \\
If $(n-1)(n)(n+1)=a^k$ then by adjacent coprimality $n=b^k,n^2-1=c^k=(b^2)^k-1$

6 \\
StackExchange. Kind of induct recurse deduce $\boxed{\frac{2^{n-1}+3+2(-1)^{n-1}}{3}}$

7 \\
Factor as $(n-1)(1+n+\dots+n^{a-1})=(n-1)(1+n+\dots+n^{\text{gcd}(a,b)-1})(1+n^{\text{gcd}(a,b)}+\dots+n^{a-\text{gcd}(a,b)})$ and similarly and for example deduce coprimality of those $2$ terms just like polynomial division like repeatedly subtract off multiplied large shift copies until one is left with a smaller quantity residue and iterate Euclidean algorithm down to $1$.

8 \\
The minimum necessarily is surrounded by copies of itself and expanding outward in a taxicab distance shape of deductions it follows.

9 \\
Putnam. $4\left(\frac{a(a+1)}{2}+\frac{b(b+1)}{2}\right)=(a-b)^2+(a+b+1)^2$ and $\frac{a^2+b^2-1}{4}=\frac{1}{2}\left(\frac{a+b-1}{2}\right)\left(\frac{a+b-1}{2}+1\right)+\frac{1}{2}\left(\frac{a-b-1}{2}\right)\left(\frac{a-b-1}{2}+1\right)$

\newpage

\textbf{Calculus}

1 \\
To be quite honest I am not sure if I misread this task, if the $x$ in the integral was supposed to be a $t$ or what practice taking the derivative of an integral with respect to $x$ but I supposed the answer may be $2\left(\int_0^{x^2}e^{-t^2}dt \right)(2x)(e^{-x^2})=\boxed{2\sqrt{\pi}e^{-x^2}x\text{ erf}(x^2)}$

2 \\
Putnam. Area and $\sin(2a)$ yields $\boxed{\frac{4}{\pi^2}}$

3 \\
Well taking the derivative with respect to $x$ one obtains $f(x)=\frac{1}{2}(f(x)+xf'(x)) \iff f(x)=xf'(x)$ i.e. $\boxed{f(x)=cx}$

4 \\
Putnam.

5 \\
Putnam.

6 \\


7 \\
StackExchange. Stirling $\boxed{\sqrt{e}}$

8 \\
Canonical.

9 \\
For example subtract off $\frac{1}{4}$ of the previous sum from the previous sum to obtain this identity.

10 \\
Putnam. Feynman Leibniz

11 \\
StackExchange. Apparently the answer is $\frac{9!8!4!1!}{(9+8+4+1+3)!}=\boxed{\frac{9!8!4!}{25!}}$ and one can utilize repeated transformations with the Beta function. However it is both natural to execute a change of integral over fixed sum plane intersections with the positive octant or rather note this is that integral $\int_A xy^9z^8w^4 dxdydzdw$ over that $3$-plane $x+y+z+w=1$

12 \\
So $\frac{1}{a}\int_0^a f(x)dx=\frac{\int_0^a \frac{f(x)}{2}f(x)dx}{\int_0^a f(x)dx} \iff \frac{2}{a}\left(\int_0^a f(x)dx \right)^2=\int_0^a (f(x))^2 dx$ and differentiating with respect to $a$ one obtains $4f(a)\left(\int_0^a f(x)dx\right)=a(f(a))^2+\left(\int_0^a (f(x))^2 dx \right)=a(f(a))^2+\frac{2}{a}\left(\int_0^a f(x)dx \right)^2$ thus $4f(a)g(a)=a(f(a))^2+\frac{2}{a}(g(a))^2$ and $g(a)=\int_0^a f(x)dx=\frac{2+\sqrt{2}}{2}af(a)$ whence again we obtain $f(a)=\frac{2+\sqrt{2}}{2}(f(a)+af'(a))$ and $\boxed{f(x)=cx^{1-\sqrt{2}}}$

\newpage

\textbf{Functional Equations}

1 \\
StackExchange. Continuous involutions $2$ fixed points surjective injective invertible symmetric around the line $y=x$ contradiction if there exists interval without fixed point on graph lying above or below this diagonal on that interval hence identity.

2 \\
$\boxed{f(x)=c\ln(x)}$ via the usual transformations.

3 \\
StackExchange. $f(x+y)+1=(f(x)+1)(f(y)+1)$ so $g(x+y)=g(x)g(y)$ is Cauchy $g(x)=x^c$ whence $\boxed{f(x)=x^c-1}$

4 \\
Probably a bounding contradiction otherwise.

5 \\
Not to be a Dr. Sonnhard Graubner but Wolfram Alpha yields $\boxed{f(x)=e^{c2^{\frac{\ln(\ln(x))}{\ln(3)}}},g(x)=-e^{c3^{\frac{\ln(\ln(x))}{\ln(2)}}}}$ manipulate $c$s.

6 \\
$\boxed{\text{No}}$ StackExchange, Rice, Schweizer, Sklar.

7 \\
I don't know my initial first thought is either to consider an easier problem like $f(x)+f(2x)=0$ or try and map this in to a statement on the complex unit circle.

8 \\
Loh.

\newpage

\textbf{Inequalities}

1 \\
$\boxed{\text{No}}$ indeed $\frac{1}{1}>\frac{18}{19},\frac{1}{19}>\frac{0}{1},\frac{18}{20}>\frac{2}{20}$ e.g.

2 \\
Flip flop and smooth to the $a_i=1$ equality case.

3 \\
Cauchy-Schwarz or smoothing e.g. to the identity permutation $\sum a_i^2$ case.

4 \\
Global minimum forces equality of neighbours and taxicab distance expansion deductions.

5 \\
Canonical.

6 \\
I think if $f'(x)$ is not bounded and $f(x)$ is bounded is probably the easiest contradiction but there perhaps exist super canonical inequalities here about limits of suprema e.g.

7 \\


\newpage

\textbf{Convergence}

1 \\
Say integral comparison works directly for these $\boxed{\text{Diverge, Diverge, Diverge, Diverge}}$

2 \\
Euler infinite tetration for $e^{-e}\le x\le e^{\frac{1}{e}}$ one obtains $y=x^y$ and bound $\boxed{e}$

3 \\
Well for example say given convergence or Ramanujan style upper bounding convergence argumentation one obtains $x=\sqrt{-\frac{2}{9}+x}$ which has solutions $\frac{1}{3},\boxed{\frac{2}{3}}$ as the sequence begins at $0.34$ and is strictly increasing by an inequality induction argument.

4 \\
Partial fractional re composition in comparison with $\sum \frac{1}{n^2}$ which converges and $\boxed{\text{Yes}}$ the other direction similar contradiction with harmonic series divergence and strict inequalities for sufficiently large of the consecutive $8$ sets.

5 \\
A Problem Seminar. Consider maximum of previous $2$ sequence.

6 \\
Erdos-Szekeres is an algorithm produces length for any length e.g. to each element associate a pair maximum increasing and decreasing chain which terminates there etc. etc. pigeonhole principle.

7 \\
$\boxed{\text{No}}$ the harmonic series diverges and this summation includes $a_1(1+\frac{1}{2}+\dots)$. $\boxed{\text{Yes}}$ StackExchange, AM-GM, Stirling.

8 \\
Probably some direct inequality algebraic manipulation hunting.

9 \\


\newpage

\textbf{Recursions}

What's going on is that $\frac{1}{89}=\frac{1}{10^2-10-1}=\frac{1}{x^2-x-1}$ is the Fibonacci characteristic polynomial recursion evaluated at a certain point with certain decimal precision leads to this in a say order series expansion.

1 \\
It certainly is strictly increasing and if it approached a finite limit one would obtain the contradiction passing the limit that $L=L+\frac{1}{L}$

2 \\
$101^2-2\cdot 101=101\cdot 99=\boxed{9999}$

3 \\
$1,2$ decompositions argumentation case on whether or not the middle dude is covered by a $2$ or not i.e. is partitioned in to $2$ copies of the $n$ or $n+1$ case.

4 \\
Sequence A000930 on the OEIS. $\boxed{277}$

5 \\
A sort of standard AIME type solution would involve counting the number of strings of each length terminating in $!A,A,AA$ like: \\
$
\begin{matrix}
2 & 1 & 0 \\
6 & 2 & 1 \\
18 & 6 & 2 \\
52 & 18 & 6 \\
152 & 52 & 18 \\
444 & 152 & 52
\end{matrix}
$ \\
Thus one obtains $\boxed{\frac{8}{9}}$

6 \\
$\boxed{\frac{1991009}{1991011}}$

7 \\
VTRMC. Fibonacci recurrence deduction and induction bound works.

8 \\
It just might have something to do with $e$

\newpage

\textbf{Linear Algebra}

1 \\
Pennsylvania.
$
\begin{vmatrix}
A^{-1} & 0 \\
-C & A
\end{vmatrix}
\begin{vmatrix}
A & B \\
C & D
\end{vmatrix}
=
\begin{vmatrix}
I & A^{-1}B \\
0 & AD-CB
\end{vmatrix}
$

2 \\
In the integer case one can deduce this via parity and repeated division argumentation and here one can suitably precisely approximate with rationals and it extends.

3 \\
Stack Exchange. Follows via complex conjugacy determinant argumentation as $I+A^2=(I+iA)(I-iA)$

4 \\
Stanford.
$
\begin{vmatrix}
I & X \\
-Y & I
\end{vmatrix}
=
\begin{vmatrix}
I & 0 \\
-Y & I+YX
\end{vmatrix}
\begin{vmatrix}
I & X \\
0 & I 
\end{vmatrix}
=
\begin{vmatrix}
I+XY & X \\
0 & I
\end{vmatrix}
\begin{vmatrix}
I & 0 \\
-Y & I
\end{vmatrix}
$

5 \\
It follows as the condition means $I-A$ is strictly diagonally dominant, recall the proof of that theorem being a contradiction argument supposing the existence of a linear dependence instance on magnitude.

6 \\
StackExchange. Suppose that $A=UV$ is a rank decomposition. We want $UVBUV=ABA=A=UV$ hence it suffices to find a $B$ such that $VBU=I$

7 \\
Russia, Problems From The Book, for further reading Bhargava. Binomial coefficient $\binom{m}{k}$ as a linear combination of $m^k,\binom{m}{k-1},\dots,\binom{m}{0}$ and transformation via the usual row operations from
$
\begin{vmatrix}
\binom{x_1}{0} & \binom{x_2}{0} & \dots & \binom{x_n}{0} \\
\binom{x_1}{1} & \binom{x_2}{1} & \dots & \binom{x_n}{1} \\
\dots & \dots & \dots & \dots \\
\binom{x_1}{n-1} & \binom{x_2}{n-1} & \dots & \binom{x_n}{n-1}
\end{vmatrix}
$
to
$
\begin{vmatrix}
1 & 1 & \dots & 1 \\
x_1 & x_2 & \dots & x_n \\
\dots & \dots & \dots & \dots \\
x_1^{n-1} & x_2^{n-1} & \dots & x_n^{n-1}
\end{vmatrix}
$

8 \\
Problems From The Book. $\boxed{1}$

\newpage

\textbf{Combinatorics}

1 \\
Monovariant algorithm e.g. if some person knows more than $3$ people in their room move that person over to the other room and do this repeatedly decreasing until target state.

2 \\
A Problem Seminar. Moser spindle for $4$ colours.

3 \\
A Problem Seminar and Putnam And Beyond. If not then without loss of generality red does not attain $r$ and blue does not attain $b$ with $r \le b$ means a circle around a blue point with radius $b$ must all be red but then there exists a segment pair there with distance $r$ contradiction.

4 \\
A Problem Seminar. Without loss of generality the coins weigh $1$ and $0$. Weigh $A,B$ if the same done just weigh each other $1$ separately otherwise weigh $A,C$ again if the same done as we deduce $B$ from earlier memory and can weigh $D$ alone but if different we are also done as we deduce that $B$ and $C$ are the same and can deduce everything from weighing $B,C,D$

5 \\
ISL.

6 \\
$\boxed{n!}$ one can certainly just expand and re compress summing over for each coefficient combinatorial usual identities manipulations.

7 \\
Pascal matrix each entry multiplied with $(-1)^{a+b}$

8 \\
A Problem Seminar.

9 \\
Iran, Sutner.

10 \\
StackExchange. Induction on the number of vertices and handling of inversion in induced subgraphs and parities.

\newpage

\textbf{Integer Polynomials}

1 \\
e.g. if $P(0)=c_0=p$ then plugging in enough multiples of $p$ will always yield a multiple of $p$ and if this is always $p$ then it must be constant contradiction.

2 \\
The $c_0$ terms cancel and $(a-b)|(a^n-b^n)$ in all the other terms.

3 \\
Canonical.

4 \\
Other file $P(1)$ gives the sum of the coefficients and then for example taking $10^n>P(1)$ one has that $P(10^n)$ will be a read off of the coefficients.

5 \\
It follows from distinct and divisibility that $(b-a)|(c-b),(c-b)|(a-c),(a-c)|(b-a)$ whence $|b-a| \le |c-b| \le |a-c| \le |b-a|$ whence $|b-a|=|c-b|=|a-c|$ contradiction.

6 \\
5 above.

7 \\
Algebraic manipulation works.

8 \\
A Problem Seminar. Note that incorrectly for example $x^p-x \equiv 0 \pmod{p}$ which gives $x^2(x-1)^2(x^4-1)^2$ has inadequate degree $12$, where the degree of $x$ need only be that of the maximum prime power factor of $n$, but in general one can take, utilizing the fact from 3, $x(x-1)(x-2)\dots (x-(m-1))$ where $m$ is the minimum such that $n|m!$ in this case $x(x-1)(x-2)\dots (x-9)$ or any such product of $m$ consecutive integers for that matter.

9 \\
Putnam.

10 \\
Putnam.

11 \\
Putnam.

\newpage

\textbf{Probability}

1 \\
$\boxed{\text{Yes}}$ for example by flipping until the binary string generated thus far disagrees with $p$ where the event occurs if the binary string has been determined to be $<p$ and does not occur if $>p$

2 \\
Sicherman dice, cyclotomic polynomials. Generating functions so $(x+x^2+x^3+x^4+x^5+x^6)(x+x^2+x^3+x^4+x^5+x^6)=(x+x^2+x^3+x^4+x^5+x^6)^2=x^2(1+x)^2(1+x^2+x^4)^2=x^2(1+x)^2(1+x+x^2)^2(1-x+x^2)^2$ and e.g. one needs to produce terms with the sums of the coefficients e.g. $x=1$ evaluations being $6$ for the $6$ sides on each dice whence one obtains the unique other positive decomposition of $(x+2x^2+2x^3+x^4)(x+x^3+x^4+x^5+x^6+x^8)$ is dice with sides $1,2,2,3,3,4$ and $1,3,4,5,6,8$

3 \\
Linearity of expectation e.g. again a permutation statistic of incidence where the $P$ for each house is $\frac{1}{n}$ and thus over all $n$ houses is $\boxed{1}$

4 \\
There certainly exists a very natural differential equations based solution or note that $E[X]=\sum_{k=0}^{\infty} P[\text{sum of first }k\text{ values}<1]=1+1+\frac{1}{2!}+\frac{1}{3!}+\dots=\boxed{e}$ by the usual simplex volume formula.

5 \\
Stochastic processes expected hitting times various conditions random walks desiderata in this case observation and induction work $\boxed{mn}$

6 \\
Algebraically with a decision tree summation of depths argumentation but I think it might be that the aforementioned strategy is optimal with respect to expected stopping time and that in this case also the evaluation is the equality case or something. We smooth to halting as quickly as is possible in rational approximation via powers of $2$.

7 \\
Robbins, permutation statistics, sequence A002464 in the OEIS. $\boxed{\frac{1}{e^2}}$

\newpage

\textbf{Bare Hands}

Wow Po, incredible, educational, instructive, and illuminating content get the kiddos thinking about some real mathy puzzley open task.

1 \\
A Problem Seminar. 

2 \\
A Problem Seminar. $\boxed{\frac{1}{x-y}}$

3 \\
Karatsuba multiplication say to compute $ac-bd,ad+bc$ one can compute $ac,bd,(a+b)(c+d)=ac+ad+bc+bd$ and note $ad+bc=(a+b)(c+d)-ac-bd$

4 \\
$\boxed{\frac{15}{2}}$

5 \\
Other file Pick and integer modulo $8$ computations.

6 \\
Putnam. Inductive construction $1$ liner.

\newpage

\textbf{Geometry}

1 \\
Pick a line and $2$ points on the curve note that in $\pi$ rotation the quantity $PA-PB$ flip flops and hence by intermediate value there exists a direction in which the induced lengths are the same e.g. the points' midpoint is $P$

2 \\
Putnam. And I quote, let $X'$ be the intersection of $AC$ and $BD$. Take $A',C'$ so that $XA'X'C'$ is a parallelogram. Similarly take $B',D'$ so that $XB'X'D'$ is a parallelogram. Then both $A'C'$ and $B'D'$ have their midpoint at the midpoint of $XX'$. Hence $A'B'C'D'$ is a parallelogram.

3 \\
A Problem Seminar. The key ideas about manipulating lines and smoothing so e.g. for each direction one can produce a line in that direction bisecting the area of a target region and e.g. here we can smooth along the lines which produce $\frac{1}{6}$ wedges until it produces those in both directions at which point we can do this again and to deal with the non concurrence issue now we can note a continuous function in terms of an orientation and note that this orientation will appear again for $\theta+\pi$ and hence was $0$ somewhere along that interval whence the desired.

4 \\
A Problem Seminar. And I quote, draw the $3$ lines a la 3. The point where they meet has this property since any line through it has $2$ wedges $+$ some more on each side of it.

5 \\
A Problem Seminar. Imitate 6 below.

6 \\
A Problem Seminar or Problems From The Book for example some sorted greedy row by row formation construction creation and the resultant algebraic inequalities contradiction argumentation if not satisfying.

7 \\
Law Of Sines if $POA=a,POB=b,PXO=c$, then $\frac{PX}{\sin(a)}=\frac{PO}{\sin(c)},\frac{PY}{\sin(b)}=\frac{PO}{\sin(\pi-(a+b)-c)}$ whence we are looking to minimize $\frac{1}{\sin(c)\sin(\pi-(a+b)-c)}$ at $\boxed{c=\frac{\pi-(a+b)}{2}}$

8 \\
A Problem Seminar. And I quote, if there were no disc of radius $r$ then the strips of width $r$ along the sides would exhaust the area, etc.

9 \\
A Problem Seminar. And I quote, cut the triangle in $2$ by dropping the altitude and then use induction.

10 \\
Putnam. $\boxed{a^2+b^2}$ without loss of generality let the ellipse be $\frac{x^2}{a^2}+\frac{y^2}{b^2}=1$ and so on and so on.
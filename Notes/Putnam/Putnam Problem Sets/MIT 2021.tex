\Large

\textbf{1}

1 \\
In each match $1$ player is eliminated and thus there will be $\boxed{214,10^{100}-1}$ matches played in all to eliminate down to the final $1$ tournament winner.

2 \\
(a) $\boxed{\text{Yes}}$ symmetric plane parallel cuts e.g. center is origin and $xy,yz,xz$ planes. \\
(b) $\boxed{\frac{n^3+3n^2+8n}{6}}$ \\
(c) Potential programmatic computation of low $n$ cases is viable at the very least...

3 \\
Peter Winkler. $\boxed{\text{No}}$ the center cubelet requires to be split by cuts adjacent to its $6>5$ faces and these can not happen simultaneously.

4 \\
(a) Putnam Notes. $\boxed{2499}$ \\
(b) Cancel and exponentiate $\boxed{6}$

5 \\
One can often bash brute force these with essentially truth tables or try low $n$ cases if there is some deduction induction one fails to on sight. \\
(a) $\boxed{\text{Knight,Knave}}$ \\
(b) $\boxed{\text{Knight,Knight}}$ \\
(c) $\boxed{\text{Knave,Knight,Knave}}$ \\
(d) $\boxed{\text{Knave,Normal,Knight}}$

6 \\
AM-GM $\boxed{2,2\sqrt{3}}$

7 \\
(a) Lucas if and only if the binary $1$ bits in the binary expansion of $k$ are a subset of the $1$ bits of $n$ \\
(b) Kummer the number of carries when adding $k$ and $n-k$ in base $p$

8 \\
(a) Otherwise we can clear out all denominators and thus produce what isomorphs in to a polynomial with integer coefficients and necessarily must have a primitive $p$-th root of unity as a zero e.g. we obtain $a_{p-1} w^{p-1}+a_{p-2} w^{p-2}+\dots +a_1 w+a_0=0$ but the minimal polynomial of $w$ is, by cyclotomic polynomial theory, $1+x+x^2+\dots+x^{p-1}$ whence the conclusion. \\
(b) IMO.

9 \\
All but the powers of $2$ by factoring say sum of arithmetic progression sequence.

10 \\
(a) $\boxed{e^{\frac{1}{e}}}$ at $x=e$ as maximizing $x^{\frac{1}{x}}$ is equivalent with maximizing its logarithm which is $\frac{1}{x}\ln(x)$ has derivative $0=\frac{1-\ln(x)}{x^2} \to x=e$ \\
(b) $\boxed{e^{\pi}}$ by taking the logarithm of both sides and considering the function $\frac{\ln(x)}{x}$

11 \\
$\boxed{\text{Yes}}$

12 \\
Putnam. Modulo $8$ computations. In $3$ dimensions one can consider importantly on a cube $(1,0,0),(0,1,0),(0,0,1)$

13 \\
It would seem that it might be $\boxed{\text{True}}$ and if so that perhaps one could utilize series to prove this.

\newpage

\textbf{2}

14 \\
$\boxed{\text{Yes}}$ for example via parallel congruent rectangles.

15 \\
$\boxed{\text{Yes}}$ in a direction not induced by the point set one may continuously shift the line and obtain any desired splitting.

16 \\
(a) $\boxed{(-1)^{n+1}}$ Putnam Notes \\
(b) $\boxed{F_{n+2}-1}$

17 \\
$\boxed{2^{n-1}}$ by choosing where in $1+1+\dots+1$ to erase $+$ signs e.g.

18 \\
The sum of all the digits used in writing down the numbers from $000000000$ to $999999999$ would be $9 \cdot 1000000000 \cdot \frac{9}{2}$ by the averaging principle and so we only need to add $1$ more to that for $1000000000$ and obtain $\boxed{40500000001}$

19 \\
$\boxed{10}$ one notes the requisite condition in terms of blue and red balls algebraically but the minimum would occur if conveniently it were the case that there exists $1$ red ball and in this case the computation is isomorphic with computing the probability that one does not select the red ball i.e. step by step $1 - \frac{N-1}{N} \cdot \frac{N-2}{N-1} \cdot \frac{N-3}{N-2} \cdot \frac{N-4}{N-3} \cdot \frac{N-5}{N-4} = 1 - \frac{N-5}{N}$ telescoping.

20 \\
$\boxed{\text{Latter}}$ via for example explicit computation of number of such subsets of size $13$ out of viable subsets of size $13$. But to determine the inequality that e.g. this stronger information leads to a stronger update in favour of the hypothesis that she holds $\ge 2$ aces one could for example order the suits Spades, Hearts, Clubs, Diamonds and consider given the information the probability of hitting these later elements in the remaining set versus the follow on question in the former case something like: what is the earliest ace you hold? Where if she now replies Spades we are in this case but in the other cases we are in fact able to exclude earlier aces which exclude them from the denominator calculus but significantly decrease the numerator calculus and therefore are weaker.

21 \\
Otherwise one would obtain that there existed a prime factor $p|d \to p|(n^2+1)$ with $p \equiv 3 \pmod{4}$ whence $-1$ is a quadratic residue modulo $p$ but computing the Legendre symbol directly one obtains $\left(\frac{-1}{p}\right) \equiv (-1)^{\frac{p-1}{2}} \equiv -1 \pmod{p}$ contradiction.

22 \\
One can conceive of a number of metrics which might have to do with the information theoretic transfer from the wall to the customer but if the viewing angle is to be maximized then it suffices to note that for each angle the locus forming that angle is a sub section of a circle so in particular the maximal possible angle will occur when this circle first becomes tangent with the floor i.e. in this case the power of the point yields $d(d+h)=x^2 \to x=\boxed{\sqrt{d(d+h)}}$

23 \\
(a) Putnam Notes. Error term bounding. \\
(b) Algebraically one ought to note that inductively e.g. one obtains $F_1^2+F_2^2+\dots +F_n^2 = F_n F_{n+1}$ and then it is more readily manipulable that the right hand side is equivalent via the aforementioned Binet generating function characteristic polynomial result above.

24 \\
$\boxed{\text{Yes}}$ A Fibonacci-Like Sequence Of Composite Numbers.

25 \\
$\boxed{\lfloor n\phi \rfloor}$ sequence A000201 in the OEIS.

26 \\
$\boxed{5}$ wiggle the pieces around to form $5$ copies of the center square.

27 \\
Peter Winkler Mathematical Puzzles.

\newpage

\textbf{3}

28 \\
Square some things and wiggle some things around Trivial Inequality.

29 \\
$\text{Speed}=\frac{\text{Distance}}{\text{Time}}$ so $\frac{2}{\frac{1}{x}+\frac{1}{y}}$ perhaps this is known as a harmonic mean.

30 \\
$\frac{a}{x}+\frac{a}{y}=1 \to a(x+y)=xy \to a=\boxed{\frac{xy}{x+y}}=\frac{1}{\frac{x+y}{xy}}=\frac{1}{\frac{1}{x}+\frac{1}{y}}$ this is probably linear or inversive to a mean known as the harmonic mean though I can't be quite sure to be honest.

31 \\
Following the earlier task with the wall and the viewing angle one may be able to construct segments of lengths $x,y$ on a line and then an arbitrary circle through those $2$ points and a tangent to that circle from the reference point. Another idea depending on just which instruments and operations are permitted would involve the ability to bisect a line segment and then construct a circle with that line segment as diameter which can then further be used with line segment transfer to produce say right triangles as needed to satisfy $a^2+b^2=c^2$ or $a^2-b^2=c^2$ type length constructions. Now with these supported one can produce a segment of length $\sqrt{x^2+y^2}$ as well as $\sqrt{\frac{x^2+y^2-xy}{2}}$ and thus $\sqrt{x^2+y^2-xy}$ and thus $\sqrt{xy}$ as desired for example with the construction of a $\frac{\pi}{3}$ angle then a $1:1:\sqrt{2}$ right triangle and then using this difference of squares production technique.

32 \\
$\left(x-\frac{1}{2} \right)^2+\left(y-\frac{1}{2} \right)^2=\frac{1}{2}$ thus $\boxed{\frac{1+\sqrt{2}}{2}}$

33 \\
(a) Power Mean. \\
(b) $\boxed{\text{max}(x,y),\sqrt{xy},\text{min}(x,y)}$

34 \\
Arithmetic-Geometric Mean.

35 \\
(a) $2$ \\
(b) $e^{-e} < x < e^{\frac{1}{e}}$ \\
(c) Sequence A033917 in the OEIS.

36 \\
All points have the same invariant sum $a+b+c$ by for example the sum of the areas extends to any equilateral polygon known as Viviani.

37 \\
Alice can choose to obtain $a_1+a_3+\dots +a_{2n-1}$ or $a_2+a_4+\dots +a_{2n}$ and these partition the set leaving Bob with the other complement.

38 \\
$\boxed{p \equiv 1,4 \pmod{5} \to p|F_{p-1},p \equiv 2,3 \pmod{5} \to p|F_{p+1}}$ as $F_{p-1}F_{p+1}=F_p^2-(-1)^{p-1}=F_p^2-1$ and it suffices to show that $F_p^2 \equiv 1 \pmod{p}$ but this follows as StackExchange Binet $F_p^2=\frac{1}{5 \cdot 2^{p-1}} \left(\sum_{j=0}^{\frac{p-1}{2}}\binom{p}{2j}3^{p-2j}5^j \right) + \frac{2}{5} \equiv \frac{3^p}{5 \cdot 2^{p-1}}+\frac{2}{5} \equiv \frac{3}{5}+\frac{2}{5} \equiv 1 \pmod{p}$

39 \\
(a) One can obtain $(n-1)^2+3(n-1)=(n-1)(n+2)=n^2+n-2>n^2$ for $n \ge 3$ by taking a simple $(n-1) \times (n-1)$ grid partition and cutting each main diagonal square in to a $2 \times 2$ \\
(b) Some intervals argumentation. \\
(c) It isomorphs in to a logical combinatorics computer sciency type question by considering such configurations as underlying maximal grid configurations with some segments erased to satisfy the desideratum.

\newpage

\textbf{4}

40 \\
The given and factoring $20+10+9=39=3\cdot 13$ leads to inspection $8+8+4,8+1+1,4+4+1$ thus it was $\boxed{C}$ who came in second in the geometry test.

41 \\
Modulo $2$ works.

42 \\
Reflect a segment and the shortest distance between $2$ points is a line.

43 \\
$\boxed{n \in \mathbb{Z}_{>0}}$ indeed this is a trolly task the structure underlying this is the sum of a row vector and a column vector in this case $[8,2,10,0]$ and $[5,0,1,4]$ with the constant norm shift of $1$. So the simplest antimagic square is: \\
$
\begin{bmatrix}
1 & 2 & \dots & n \\
n+1 & n+2 & \dots & 2n \\
\dots & \dots & \dots & \dots \\
n^2-n+1 & n^2-n+2 & \dots & n^2
\end{bmatrix}
$

44 \\
Well it starts out with $1$ plane region at $0$ lines and then each line adds $n$ new regions thus $1+1+2+\dots +n=\frac{n(n+1)}{2}+1=\boxed{\frac{n^2+n+2}{2}}$

45 \\
It is the set of non negative integers with base $3$ strings consisting of only $0$s and $1$s so convert $1000000$ to binary and then interpret that binary string as a ternary string to obtain $11110100001001000000_3=\boxed{1726672221}$

46 \\
(a) Counting in $2$ ways the number of squares covered. \\
(b) I think one can consider complex numbers powers of roots of unity or a discrete association with $x+y \pmod{a}$ of toroidal diagonals wherein the multiplicities of each residue covered by a board are all the same and thus the multiplicities of each residue over the whole checkerboard can be thus determined and it follows. \\
(c) It generalizes.

47 \\
Fourteen Proofs Of A Result About Tiling A Rectangle.

48 \\
$\boxed{\text{No}}$ assume that it is and let $n$ be the number of quadrilaterals. Then the total sum of all internal angles must be $2\pi n$ however each of the non convex angles must be in the interior of the polygon hence the sum of angles around these vertices must also be $2\pi n$ contradiction as there are also the initial angles at the vertices of the polygon which must be accounted for in the angle calculus.

49 \\
(a)  \\
(b) 

50 \\
Sparse Polynomial Square.

\newpage

\textbf{5}

51 \\
$\boxed{\approx 734}$

52 \\
(a) $\boxed{11}$ by decision tree depth ask bit by bit. \\
(b) $\boxed{15}$ see Hamming code and Three Thresholds For A Liar.

53 \\
Counting in $2$ ways parity pairing off each must appear $1$ time on the main diagonal.

54 \\
$\boxed{6210001000}$

55 \\
$\boxed{\frac{2\pi}{3}-\frac{\sqrt{3}}{2}}$

56 \\
(a) Again imitating as earlier for any direction not induced by the original point set one can continuously shift a line to partition the point set as desired and what is a line but a degenerate circle arbitrarily large if forced to be finite. \\
(b) APMO general.

57 \\
(a) Suppose not then $k$ must have a prime factor $p|(n^4+n^3+n^2+n+1)$ which is also not congruent to $0,1 \pmod{5}$ and thus $n^4+n^3+n^2+n+1 \equiv 0 \pmod{p}$ but then so is e.g. $n^9+n^8+\dots +1 \equiv 0 \pmod{p}$ by multiplying with $n^5$ and so on and so on thus as we also know $n$ can not be $\equiv 1 \pmod{p}$ as that would mean $1+1+1+1+1 \equiv 5 \equiv 0 \pmod{p}$ and similarly $n$ can not be $\equiv 0 \pmod{p}$ and thus we may divide out in the following by $n$ and $n-1$ whilst preserving the $0$ on the right hand side. We deduce from $p|(n^{p-1}-1)$ that $n^{p-2}+n^{p-3}+\dots +1 \equiv 0 \pmod{p}$. Now $p \equiv 2,3,4 \pmod{5}$ means that $p-2 \equiv 0,1,2 \pmod{5}$ and we deduce that one of $1,1+n,1+n+n^2$ is $\equiv 0 \pmod{p}$. But in fact again cancel via subtraction in the latter case would yield the $1+n$ case which itself again yields the $1$ case contradiction. \\
(b) Say there are finitely many such primes $p_1,p_2,\dots,p_n$ then consider for the prime factorization for $n=5p_1p_2\dots p_n+1$ it must contain a new one by construction contradiction.

58 \\
$\boxed{36\pi}=\frac{4}{3}\pi(3)^3$ Napkin ring problem.

59 \\
Napoleon.

60 \\
Polynomials Associated With Reciprocation.

61 \\
Conway's Soldiers.

62 \\
Robin Chapman, Noam Elkies, et al. Deja News usenet thread, Hasse-Minkowski theory of rational quadratic forms.

63 \\


\newpage

\textbf{6}

64 \\
It would be a $6$ with itself reflected over leftwards.

65 \\
(a) Place $0,1,2,\dots,9$ from each bottle respectively and the error term will give away which bottle is faulty. \\
(b) Sure in fact for $11$ bottles placing $285,433,510,550,570,581,587,590,592,593,594$ works see sequence A005318 in the OEIS Conway-Guy each subset has distinct sum and thus is information theoretically adequate to distinguish the requisite subset of bottles from a single weighing.

66 \\
(a) $\boxed{3}$ \\
(b) $\boxed{\frac{3^k-1}{2}}$

67 \\
(a) $x^2-x=x(x-1)$ is a set of $2$ adjacent integers and thus $1$ must be $\equiv 0 \pmod{2}$ \\
(b) $x^3-x=(x-1)x(x+1)$ so $1$ must be $\equiv 0 \pmod{2}$ and $1$ must be $\equiv 0 \pmod{3}$ \\
(c) See sequence A027760 in the OEIS it is the product of primes $p$ such that $(p-1)|(n+1)$

68 \\
(a) $f'=f$ has general solution $ce^t$ in this case $e^t$ thus $e^1=\boxed{e}$ \\
(b) $f'=f^2$ has general solution $\frac{1}{c-t}$ in this case $\frac{1}{1-t}$ so he/she will be at the singularity of asymptotic undefined $\infty$

69 \\
$\boxed{2^{n+1}-1}$ Lagrange Interpolation

70 \\
If $n$ is even done otherwise $n=2k-1$ and $n^4+4^n=(n^2+2^n-n2^k)(n^2+2^n+n2^k)$

71 \\
Van Aubel. Coordinate bash works.

72 \\
$\boxed{\binom{n-1}{0}+\binom{n-1}{1}+\binom{n-1}{2}+\binom{n-1}{3}+\binom{n-1}{4}}$ see sequence A000127 in the OEIS.

73 \\
(a) Maybe induction note the existence of pentagram like configurations. \\
(b) 

74 \\
The ratio between consecutive length sizes can be resolved to be $7:3$. Total distance is $10$, perhaps physics interpretation, conservations, perhaps changing reference frames makes this computation much easier.

75 \\


76 \\


\newpage

\textbf{7}

77 \\
(a) I can't be quite sure I comprehend the task if we suppose that we know the weights will be integers then it suffices to be able to produce $2,4,\dots,62$ in order to execute comparisons at which point the set of $5$ weights $2,4,8,16,32$ suffices or $6$ under some other interpretation having to do with logarithms base $2$. \\
(b) I can't be quite sure again though I suspect the intention is to invoke distinct deltas a la unique base $3$ representations using $-1,0,1$ coefficients at which point this probably has more to do with the logarithm base $3$ of $2\cdot 40+1=81$ is $4$ \\
(c) Again, this task statement lacks technical clarity and precision. In my understanding of gold chains, when one cuts a link say we can split the chain in to lengths $7,1,15$ with one operation... but in terms of simple theoretical partitions which have subsets generating all of the sums one considers $1,2,4,8,8$ would be $4$ cuts.

78 \\
Combinatorial Enumeration.

79 \\
Moving from $n=1$ to $n=2$ both groups of people contain the empty set.

80 \\
It suffices to note that $\frac{\frac{1}{2}+\frac{1}{3}}{1-\frac{1}{2}\cdot \frac{1}{3}}=1$ by $\tan(a+b)=\frac{\tan(a)+\tan(b)}{1-\tan(a)\tan(b)}$

81 \\
(a) $2^n-1=1+2+4+\dots+2^{n-1}$ has $n$ terms and if not prime then this permits factorization via cyclotomic type decomposition if $n=ab$ then $(1+2+\dots+2^{a-1})(1+2^a+2^{2a}+\dots+2^{(b-1)a})$ \\
(b) Otherwise $(2^b+1)|((2^b)^a+1)$ where $a$ is a non trivial odd factor of $n$

82 \\
(a) $\boxed{\sqrt{19}}$ \\
(b) Symmetry of simplex. \\
(c) $(n+1)(d^4+a_0^4+a_1^4+\dots+a_n^4)=(d^2+a_0^2+a_1^2+\dots+a_n^2)^2$ \\
(d) Symmetry of simplex.

83 \\
$\boxed{4}$

84 \\
Bertrand Postulate or maximum power of $2$ e.g.

85 \\
(a) Common knowledge. \\
(b) Common knowledge.

86 \\
See sequence A087910 in the OEIS. Sydney University Mathematical Society Problems $2$-adic logarithms.

87 \\


88 \\


\newpage

\textbf{8}

89 \\
That is to say that Pat won $3/4$ of Stacy games and $3/5$ of her own games and thus $\boxed{\text{Pat}}$ served first.

90 \\
$2=44-42$ or if we are allowed rather than turning an hourglass upside down if we are instead allowed to mark the hourglass and make assumptions about sand flow than those sorts of solutions return in to existence.

91 \\
These $\frac{1}{y^2}$ terms are not well defined over the region of integration where $y=0$ i.e. one can not divide by $0$ as such.

92 \\
$24,24,24, \to 12,12,48 \to 6,42,24 \to \boxed{39,21,12}$

93 \\
$\boxed{2}$ for example shift the middle $\frac{1}{2}$ to the left and the right $\frac{3}{2}$ to the left and $1$ down.

94 \\
Fibonacci recursion $\boxed{F_{n+2}}$

95 \\
$0,1,3 \pmod{4}$

96 \\
Unfolding one obtains a $24,32,\boxed{40}$

97 \\
(a) $\boxed{\text{Yes},7}$ \\
(b) $\boxed{\text{Yes}}$ Acute Triangulations Of Polyhedra And $\mathbb{R}^n$

98 \\
Well I can't be quite sure what comparable means and this task is not well mathematically formulated but one can consider a variety of internal consistency desiderata. If this was on a finance interview I might assert something about running some ELO numbers on a real statistical data set.

99 \\
Peter Winkler Mathematical Puzzles. $\boxed{2^n|n\in \mathbb{Z}_{\ge 0}}$

100 \\
Peter Winkler Mathematical Puzzles. Create a sequence of partial sums of new vectors in the following way. Start with the vector which formerly was $[1,1,\dots,1]$ and after that repeatedly select the one which formerly had $-1$s where our vector is $1$ and $1$s where our vector is $0$ such that our partial sum is always a $\{0,1 \}$ vector. Then halt the first time we produce a partial sum we already produced or $0$ and the interval sum to $0$ is the desired subset. Indeed, because each partial sum was distinct each new vector was also distinct as by construction each new vector was uniquely based off of the partial sum vector.

\newpage

\textbf{9}

101 \\
$\boxed{\frac{\pi}{3}}$ equilateral triangle.

102 \\
$\boxed{121}_3=16$

103 \\
Sequence A006092 in the OEIS numbers beginning with letter t when spelled out in English.

104 \\
(a) Nim where turn based analysis reveals that $1,2,3,4$ are winning states and $5$ is a losing state i.e. the $\boxed{\text{first}}$ player wins by removing $1$ and repeatedly giving his opponent the $0 \pmod{5}$ state. \\
(b) Nim now by the same strategy the $\boxed{\text{second}}$ player wins by repeatedly giving the first player the $1 \pmod{5}$ state.

105 \\
We have left and right eyes not up and down eyes.

106 \\
The recurrence telescopes to $f(n+1)=(n+1)f(n)$ i.e. $\boxed{f(n)=n!}$

107 \\
Construction.

108 \\
(a) Optimization. \\
(b) Not at the moment.

109 \\
$\boxed{56}$

110 \\


\newpage

\textbf{10}

111 \\
Note the initial shape, how the boundary imposes structure, and how it would be very convenient if each of these $4$ congruent pieces in fact had the same shape as the overall initial shape with a linear scale factor of $\frac{1}{2}$. But this works indeed if we put $2$ up against the right edge with the same orientation we are left with an L block which can be cut in to symmetric halves.

112 \\
One can imitate the configuration in 26 for example producing the central square from one of the triangles precisely as is illustrated there.

113 \\
Modulo $17$ one obtains $10(x-3000000000000000)+3=\frac{3x}{2} \to x=\boxed{3529411764705882}$

114 \\
(a) Taylor $1+x+\frac{x^2}{2!}+\dots >x$ or minimize $e^x-x$ at derivative $e^x-1=0 \to x=0$ where $e^x-x=1$ \\
(b) The minimum of $a^x-x$ must be $0$ and it occurs when the derivative $a^x\ln(a)-1=0 \to a^x=\frac{1}{\ln(a)} \to a=\boxed{e^{\frac{1}{e}}}$

115 \\
(a) $1,2,3,4,6,8,9,11,13,16,18,23$ \\
(b) Putnam Notes $ab-a-b$ and $\frac{(a-1)(b-1)}{2}$ inexpressible.

116 \\
The payout matrix is given by
$
\begin{bmatrix}
2x & 3x-1 \\
3x-1 & 4x
\end{bmatrix}
$
whence one obtains the Nash Equilibrium when $(2x)(a)+(3x-1)(1-a)=(3x-1)(a)+(4x)(1-a),(2x)(b)+(3x-1)(1-b)=(3x-1)(b)+(4x)(1-b) \to a=b=\frac{x+1}{2} \to x=\boxed{3-2\sqrt{2}}$

117 \\
Something something something.

118 \\
Harmonic divergence smoothing equality center of mass threshold case see Maximum Overhang for further reading.

119 \\
Iran, Sutner.

120 \\
Iran, Sutner.

121 \\
StackExchange. Unfold in multiple ways and compute the circumradius of $O,O',O''$ to be $\boxed{\frac{\sqrt{130}}{4}}$

122 \\
This is the end of mathematics, Yufei.

\newpage

\textbf{11}

123 \\
$\boxed{\text{Yes}}$ Eulerian Path using the $2$ central odd degree vertices as starting and ending points for example crossing and then forming an Eulerian Cycle from the remainder L,R,D,L,U,R,DR,D,DL,L,UL,U,UR,DR,DL,UL,UR,R.

124 \\
(a) $\boxed{(0,2),(0,3),(1,4),(1,6),(11,12)}$ \\
(b) $\boxed{\text{Yes}}$ \\
(c) $\boxed{\text{Yes}}$ I would imagine so. \\
(d) Density. \\
(e) Chinese Remainder would force no coprime I think would be a contradiction in the equality case.

125 \\
(a) $(1+x)(1+x^2)(1+x^4)(1+x^8)\dots=1+x+x^2+x^3+\dots=\boxed{\frac{1}{1-x}}$ \\
(b) Sequence A002487 in the OEIS Stern-Brocot's diatomic series. \\
(c) Algebraic manipulations.

126 \\
Hadwiger-Nelson problem, Moser spindle, Aubrey de Grey.

127 \\
For $p|f(a)\neq \pm 1$ note $f(a+np) \equiv f(a) \equiv 0 \pmod{p}$ and can only attain the values $-p,0,p$ at most the degree times each e.g.

128 \\
$\boxed{a\in \left[0,\frac{1}{2} \right]}$ as $f(x+a)-f(x)$ will go from $f(a)$ to $-f(1-a)$ eh I think frankly induced intervals from zeroes and sign changes works actually where positive until $\frac{1}{2}$ and negative after is the motivating construction for the bound.

129 \\
Yes I mean $1$ of those cases is reality and the credence ought to be framed as follows: there exists a true delta and there is a half chance of either going up or down by the true delta so it's a $0$ EV proposition to switch.

130 \\
Tiling A Square With Similar Rectangles.

131 \\
Algorithmically undoing swooping end truncated Zeckendorf binary string representations of integers as sums of subsets of the Fibonacci numbers and the parity of the number of elements in these representations.

132 \\
A Nonimmersive C$\infty$ Mapping Having The Universal Property Of Immersions.

133 \\
Odd/Even Town. $\mathbb{Z}_2^n$ incidence vector linearly independent inner product odd-town rules dimension.

\newpage

\textbf{Abstract Algebra}

1 \\


2 \\


3 \\


4 \\


5 \\


6 \\


7 \\


8 \\


9 \\


10 \\


11 \\


12 \\


13 \\


14 \\


15 \\


16 \\


17 \\


18 \\


19 \\


20 \\


21 \\


22 \\


23 \\


24 \\


25 \\


26 \\


27 \\


28 \\


29 \\


30 \\


31 \\


32 \\


33 \\


34 \\


35 \\


36 \\


37 \\


38 \\


39 \\


40 \\


41 \\


42 \\


43 \\


44 \\


\newpage

\textbf{Analysis}

1 \\


2 \\


3 \\


4 \\


5 \\


6 \\


7 \\


8 \\


9 \\


10 \\


11 \\


12 \\


13 \\


14 \\


15 \\


16 \\


17 \\


18 \\


19 \\


20 \\


21 \\


22 \\


23 \\


24 \\


25 \\


26 \\


27 \\


28 \\


29 \\


30 \\


31 \\


32 \\


33 \\


34 \\


35 \\


36 \\


37 \\


38 \\


\newpage

\textbf{Combinatorial Configurations}

1 \\
Tiling A Rectangle With L-Tetrominoes. The condition for tilability of an $a \times b$ rectangle with L-tetrominoes is $8|(ab)$ and $a,b>1$. Indeed consider the lattice tiling associating colour $x \pmod{2}$ then each L-tetromino must cover $3$ of one colour and $1$ of the other whence a parity argumentation yields $\boxed{\text{No}}$

2 \\
The Maximum Degree Of A Random Graph.

3 \\
$\boxed{c(n)=n}$. Indeed, for each $i=1,2,\dots,n$ there exists precisely $1$ such representation using $i$ summands. This can be explicitly derived in terms of multiplicities of the floor and ceiling function of $\frac{n}{i}$ in conjunction with the $1$ gap bound.

4 \\
Roots of unity filter $(2+x+x^2)^n$ for digit sums $\equiv 0 \pmod{3}$ sum of evaluations is $\boxed{\frac{4^n+2}{3}}$ as $1+x+x^2=0$ for $w,w^2$

5 \\
$18$ dominos $18$ line crossings and $10$ lines means by pigeonhole there exists a line crossed once contradiction on parity having an odd number of squares remaining on each side.

6 \\
StackExchange $\boxed{\text{Yes}}$ indeed suppose the sets are $A_1,A_2,\dots,A_{2020}$ and $A_1=\{a_1,a_2,\dots,a_{42} \}$ so we can partition the other dudes based on which single element they intersect $A_1$ with then by pigeonhole one of the partitions must have size at least $49=\lceil \frac{2019}{42} \rceil$ so without loss of generality say it was $a_1 \in A_2,A_3,\dots,A_{50}$ and $a_1 \notin A_{2020}$ so each of those $49$ dudes have to intersect with a unique element of $A_{2020}$ but there is the contradiction as $49>42$.

7 \\
Putnam.

8 \\
Maybe Putnam $\boxed{\lfloor \frac{n+1}{2} \rfloor}$

9 \\
Putnam.

10 \\
Putnam. $\boxed{\text{Marie}}$ for example a copy cat strategy to force rows $1,2$ to be identical.

11 \\
Putnam.

12 \\
Putnam.

13 \\
Methinks this has to do with the linearity of expectation and the configuration of say $7$ circles with centers at the vertices and center of a regular hexagon and uniformly random planar tilings therein.

14 \\
$\boxed{\text{Yes}}$ by Hall as indeed $m$ days hit $m$ teams. If $m$ days do not hit $p$ teams then it must be the case that no pair of those teams played each other on those $m$ days, thus they played each other all on those other days, thus there were at least $p-1$ other days and we deduce that $m+(p-1) \le 2n-1 \to m \le 2n-p$ as desired as the right hand side counts the number of teams which were hit on those $m$ days.

15 \\


16 \\
Putnam.

17 \\
Isomorphs in to a statement about binary matrices and frequently appears in combinatorics handouts. One solution is to note that counting in two ways incidences linearity of expectation whatever there exists a student who solved $\ge \frac{120}{200} \cdot 6=\frac{720}{200}$ problems e.g. they solved $4$ problems. Now consider that there were $240$ solves on the other $2$ problems, thus there exists some student who solved those $2$. Now these together solved all $6$ problems.

18 \\
Putnam.

19 \\
Putnam.

20 \\
I think that for equality to occur would require that in $3 \binom{11}{2}=11\left(\binom{4}{2}+\binom{4}{2}+\binom{3}{2} \right)$ one would have each of the $3$ colours precisely induces $55$ out of $55$ of the column pairs over all the rows but there is no such way under the $4,4,3$ split constraint for these multiplicities even to align properly.

21 \\
I think there exists a partial sum subsequence interval sum type solution here mapping each of $1,2,\dots,n$ in a certain fashion with the relevant interval endpoints and considering lengths.

22 \\
Putnam.

23 \\
See 124 above.

24 \\
I would imagine some induction actually works.

25 \\
Kvant, MOSP.

26 \\
StackExchange. Indicator vector $\mathbb{F}_2^n$ symmetric matrix $\text{det} M=0$ non trivial kernel non trivial $Mu=0$ vector satisfies $s_i \cdot u=0$

27 \\
Putnam.

28 \\
Non-Uniform Fisher's Inequality.

29 \\
Putnam.

30 \\
Putnam.

31 \\
Putnam.

\newpage

\textbf{Congruences And Divisibility}

1 \\


2 \\


3 \\


4 \\
Quora. It suffices to show that $\sum_{r=1}^{\lfloor \frac{2p}{3} \rfloor} \frac{1}{p} \binom{p}{r} \equiv 0 \pmod{p}$ now $\frac{1}{p} \binom{p}{r} = \frac{1}{p} \cdot \frac{p!}{r!(p-r)!}=\frac{(p-1)(p-2)\dots(p-r+1)}{r!} \equiv \frac{(-1)(-2)\dots(-(r-1))}{(1)(2)\dots(r-1)(r)} \equiv \frac{(-1)^{r-1}}{r} \pmod{p}$. The question is now transformed in to showing that $1-\frac{1}{2}+\frac{1}{3}-\dots (-1)^{\lfloor \frac{2p}{3} \rfloor +1} \frac{1}{\lfloor \frac{2p}{3} \rfloor} \equiv \left(1+\frac{1}{2}+\dots+\frac{1}{\lfloor \frac{2p}{3} \rfloor} \right) -2\left(\frac{1}{2}+\frac{1}{4}+\dots+\frac{1}{2 \lfloor \frac{p}{3} \rfloor} \right) \equiv \left(1+\frac{1}{2}+\dots+\frac{1}{\lfloor \frac{2p}{3} \rfloor} \right) -\left(1+\frac{1}{2}+\dots+\frac{1}{\lfloor \frac{p}{3} \rfloor} \right) \equiv \frac{1}{\lceil \frac{p}{3} \rceil}+\frac{1}{\lceil \frac{p}{3} \rceil +1}+\dots+\frac{1}{\lfloor \frac{2p}{3} \rfloor} \equiv 1+\frac{1}{2}+\dots+\frac{1}{p-1} \equiv 1+2+\dots+(p-1) \equiv \frac{p(p-1)}{2} \equiv 0 \pmod{p}$ where we could also simply utilize the identity $\frac{1}{r}+\frac{1}{p-r} \equiv 0 \pmod{p}$ pairing off terms in that intermediate step in order to prove the claim. It is still instructive to note that this summation can be transformed in this way via the uniqueness of inverses modulo $p$ so that these two summation sets are the same.

5 \\


6 \\


7 \\


8 \\


9 \\


10 \\
Chinese Remainder $\equiv p_1 \pmod{p_1^2}, \equiv p_2-1 \pmod{p_2^2}, \equiv p_3-2 \pmod{p_3^2}, \dots, \equiv p_k-(k-1) \pmod{p_k^2}$

11 \\


12 \\


13 \\


14 \\


15 \\


16 \\


17 \\


18 \\


19 \\
$\boxed{3}$

20 \\


21 \\


22 \\


23 \\
Putnam.

24 \\
Counting in $2$ ways exponents of prime factors.

25 \\
Probably Putnam.

26 \\


27 \\


28 \\


29 \\


30 \\


31 \\
Putnam or Putnam And Beyond factoring.

32 \\


33 \\


34 \\


35 \\


36 \\
Putnam.

37 \\
Putnam.

\newpage

\textbf{Generating Functions}

1 \\


2 \\


3 \\


4 \\


5 \\


6 \\


7 \\


8 \\


9 \\


10 \\


11 \\


12 \\


13 \\


14 \\


15 \\


16 \\


17 \\


18 \\


19 \\


20 \\


\newpage

\textbf{Independence And Uniformity}

1 \\


2 \\


3 \\


4 \\


5 \\


6 \\


7 \\


8 \\


9 \\


10 \\


11 \\


12 \\


13 \\


14 \\


15 \\


16 \\


17 \\


18 \\


19 \\


20 \\


\newpage

\textbf{Inequalities}

1 \\


2 \\


3 \\


4 \\


5 \\


6 \\


7 \\


8 \\


9 \\


10 \\


11 \\


12 \\


13 \\


14 \\


15 \\


16 \\


17 \\


18 \\


19 \\


20 \\


21 \\


22 \\


23 \\


24 \\


25 \\


26 \\


27 \\


28 \\


29 \\


\newpage

\textbf{Polynomials}

1 \\


2 \\


3 \\


4 \\


5 \\


6 \\


7 \\


8 \\


9 \\


10 \\


11 \\


12 \\


13 \\


14 \\


15 \\


16 \\


17 \\


18 \\


19 \\


20 \\


21 \\


22 \\


23 \\


24 \\


25 \\


26 \\


27 \\


28 \\


29 \\


30 \\


31 \\


32 \\


33 \\


34 \\


35 \\


36 \\


37 \\


38 \\


39 \\


40 \\


41 \\


\newpage

\textbf{Probability}

1 \\
$\boxed{\text{Yes},\frac{2000}{3}}$ switch wins with $P=\frac{2}{3}$

2 \\
Supposing that this is asking for the number of queen pairs out of all remaining viable pairs one obtains $\boxed{\frac{6}{22},\frac{5}{13},\frac{3}{7}}$

3 \\
Roots of unity filter sum of $\frac{1}{2}((1-p)+px)^n$ evaluations at $-1,1$ so $\frac{1+(1-2p)^n}{2}$

4 \\
Geometric probability $\boxed{\frac{7}{16}}$

5 \\
The expected value is the sum of the probabilities of not clearing the threshold $1$ after $n-1$ turns i.e. the sums of the volumes of simplices which is precisely $e=1+1+\frac{1}{2!}+\frac{1}{3!}+\dots$

6 \\
Geometric probability $\boxed{\frac{(b-a)^2}{b^2}}$

7 \\
Putnam. $\boxed{\frac{5-\pi}{4}}$

8 \\
Putnam. $\boxed{1-\frac{35}{12\pi^2}}$

9 \\
Putnam. $\boxed{\frac{k!(k+1)!}{(3k+1)(2k)!}}$

10 \\
Putnam. The expected value of the sum is $\int_0^1 f(t) (1-(1-t)^n) dt$

11 \\


12 \\
$\boxed{\frac{4\sqrt{2}-5}{3}}$

13 \\
$\boxed{\text{Yes}}$ flip until the binary string produced is determined to be $<a$ or $>a$ and win if $<a$.

14 \\
Putnam.

15 \\
$\boxed{\frac{4(\sqrt{2}-1)}{3}}$

16 \\
Mathematical Puzzles. $\boxed{\frac{2}{3}}$

17 \\
Putnam. In terms of equations: \\
$
\begin{matrix}
- & C<70 & C=70 & C>70 \\
D<70 & a & b & c \\
D=70 & d & e & f \\
D>70 & g & h & i
\end{matrix}
$
We are given $b+e+h, d+e+f, b+d+e$ and we compute $e+f+h=(b+e+h)+(d+e+f)-(b+d+e)$

18 \\
There is a bijection between sextuplets of distinct integers from $1$ to $36$ with septuplets of positive integers summing to $37$ e.g. via cutting $[0,37]$ in to $7$ sub intervals. But this naturally isomorphs maps bijects in to sepuplets of non negative integers summing to $30$. Then via a cyclic shifts mapping in to sets of septuplets e.g. one obtains $37-1-\frac{30}{7}=\boxed{\frac{222}{7}}$. In general integration works if not cyclic shifts argumentation and note that the expected value of the minimum of $n$ i.i.d. random variables on $[0,1]$ is $\frac{1}{n+1}$ e.g.

19 \\
(a) The expected step size is $\frac{7}{2}$ thus the density is $\boxed{\frac{2}{7}}$ \\
(b) $\boxed{\frac{1}{\sum ia_i}}$

20 \\
Putnam.

21 \\
For example as earlier it is the sum of the probabilities of not clearing the threshold by the time $t$ and so one obtains $1+\frac{\pi^1}{1!2^2}+\frac{\pi^2}{2!2^4}+\dots=\boxed{e^{\frac{\pi}{4}}}$

22 \\
Stochastic processes this is known as the expected hitting time for the level $1$ in a random walk in this case with $P=\frac{1}{2}$ and one obtains $\boxed{\infty}$ via say generating functions.

24 \\
Putnam.

25 \\
$\boxed{\text{No}}$ isomorphic with flipping up and always betting on the bottom card.

26 \\
Peter Winkler Mathematical Puzzles guaranteed maximum $\frac{2^{52}}{\binom{52}{26}}$

27 \\
Peter Winkler Mathematical Puzzles $\boxed{\text{Yes}}$ as long as Bob can randomly select a threshold value such that there is a positive probability of him selecting a threshold in any interval then he can do that and use it to decide whether or not to switch.

28 \\
Peter Winkler Mathematical Puzzles $\boxed{\text{No}}$ Alice can choose whichever value is closer to $\frac{1}{2}$ and then if this is $a$ Bob's posterior is $\frac{1}{2}$ in each of without loss of generality $[0,a],[1-a,1]$ randomly guessing for $\frac{1}{2}$

29 \\
Linearity of expectation $\boxed{2013\left(1-\left(\frac{2012}{2013} \right)^{2013} \right)}$

30 \\
HMMT. Equiprobable $\boxed{\frac{63}{2}}$

31 \\
USMCA. $\boxed{\text{Yes}}$

32 \\
Canon. $\boxed{2^n|n \in \mathbb{Z}_{\ge 0}}$

33 \\
OMO. $\boxed{\left(\frac{3}{2} \right)^n-1}$

\newpage

\textbf{Sums And Integrals}

1 \\


2 \\


3 \\


4 \\


5 \\


6 \\


7 \\


8 \\


9 \\


10 \\


11 \\


12 \\


13 \\


14 \\


15 \\


16 \\


17 \\


18 \\


19 \\


20 \\


21 \\


22 \\


23 \\


24 \\


25 \\


26 \\


27 \\


28 \\


29 \\


30 \\


31 \\


32 \\

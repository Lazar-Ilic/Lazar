Two economists are walking in a forest when they come across a pile of shit.

The first economist says to the other ``I will pay you \$100 to eat that pile of shit." The second economist takes the \$100 and eats the pile of shit.

They continue walking until they come across a second pile of shit. The second economist turns to the first and says ``I will pay you \$100 to eat that pile of shit." The first economist takes the \$100 and eats the pile of shit.

Walking a little more, the first economist looks at the second and says, ``You know, I gave you \$100 to eat shit, then you gave me back the same \$100 to eat shit. I can not help but feel like we both just ate shit for nothing." ``That is not true", responded the second economist. ``We increased the Gross Domestic Product by \$200!"

- Reddit et al. [Unclear Source] [Amerikkkan Idiocracy]

----------

Wasalu Muhammad Jaco makes trill shit! Prva liga Japan Audio Societe High Resolution Audio Input|Output Codec CRap, mang! Sikker than sik brah! ESSKEETIT! ``A self portrait of a man the wealth tortured!" Deeper than deep!

- Lazar Ilic [Not Yet An MCAT Medical College Admission Test Perfect Scorer]

----------

OK so uh might be good to skim through read some more uh like introductory economics books just to gain more exposures to more examples more random walk throughs of stuff like stock splits oh I dunno the Fed yadda increases the interest rate yadda effects downstream on this that the other economique parameter. Click through some Wikipedia maybe some Bloomberg links too. Review random other financial concepts, terms, firms, long-short equity, whatever yadda instruments the entire McCombs corpus to be sure. To be sure to be sure to be sure I mean who knew this Finance corpus is so endlessly deeply deeply deeply fascinating the entire time everyone in the Economics scene knew! There is so so so much fascinating, interesting, and potentially useful material right here wowzerz! Gotta focus focus focus tank more time and energy into Economics and Reinforcement Learning. It's amaaaaazing what 365 days can do to you ninja this time last year I was about cutting budgets and shieeet now I am finna penetrate a massive financial literature corpi and come out the rear end a Rightist internet troll telling the kiddos skadoosh skadaddle get off my lawn bro I am like a genius genius genius radical visionary genius. I am really like in awe of [snort] His creation. No denying I mean this is some real serious deep weighty meaty meat stuff right here there is real serious heavy un ironic content here to contend with and contemplate for my own noodle noggin to be sure.

Then come back here make some actually half sharp commentary maybe to really convince my readership that I know what is up and really understand the really really really critical and important stuff. Lots of information and interesting concepts here to re review.

CONTENTS



DEDICATION ix



LIST OF CONTRIBUTORS xi



INTRODUCTION
Dek Terrell and Thomas B. Fomby xiii



REMARKS BY ROBERT F. ENGLE III AND
SIR CLIVE W. J. GRANGER, KB
Given During Third Annual Advances in Econometrics
Conference at Louisiana State University, Baton
Rouge, November 5-7, 2004



GOOD IDEAS
Robert F. Engle III xix



THE CREATIVITY PROCESS
Sir Clive W. J. Granger, KB xxiii



PART I: MULTIVARIATE VOLATILITY MODELS



A FLEXIBLE DYNAMIC CORRELATION MODEL
Dirk Baur 3



A MULTIVARIATE SKEW-GARCH MODEL
Giovanni De Luca, Marc G. Genton and Nicola Loperfido 33



SEMI-PARAMETRIC MODELING OF CORRELATION DYNAMICS
Christian M. Hafner, Dick van Dijk and Philip Hans Franses 59



A MULTIVARIATE HEAVY-TAILED DISTRIBUTION FOR ARCH/GARCH RESIDUALS
Dimitris N. Politis 105



A PORTMANTEAU TEST FOR MULTIVARIATE GARCH WHEN THE CONDITIONAL MEAN IS AN ECM: THEORY AND EMPIRICAL APPLICATIONS
Chor-yiu Sin 125



PART II: HIGH FREQUENCY VOLATILITY MODELS



SAMPLING FREQUENCY AND WINDOW LENGTH TRADE-OFFS IN DATA-DRIVEN VOLATILITY ESTIMATION: APPRAISING THE ACCURACY OF ASYMPTOTIC APPROXIMATIONS
Elena Andreou and Eric Ghysels 155



MODEL-BASED MEASUREMENT OF ACTUAL VOLATILITY IN HIGH-FREQUENCY DATA
Borus Jungbacker and Siem Jan Koopman 183



NOISE REDUCED REALIZED VOLATILITY: A KALMAN FILTER APPROACH
John P. Owens and Douglas G. Steigerwald 211



PART III: UNIVARIATE VOLATILITY MODELS



MODELING THE ASYMMETRY OF STOCK MOVEMENTS USING PRICE RANGES
Ray Y. Chou 231



ON A SIMPLE TWO-STAGE CLOSED-FORM ESTIMATOR FOR A STOCHASTIC VOLATILITY IN A GENERAL LINEAR REGRESSION
Jean-Marie Dufour and Pascale Valery 259



THE STUDENT'S T DYNAMIC LINEAR REGRESSION: RE-EXAMINING VOLATILITY MODELING
Maria S. Heracleous and Aris Spanos 289



ARCH MODELS FOR MULTI-PERIOD FORECAST UNCERTAINTY: A REALITY CHECK USING A PANEL OF DENSITY FORECASTS
Kajal Lahiri and Fushang Liu 321



NECESSARY AND SUFFICIENT RESTRICTIONS FOR EXISTENCE OF A UNIQUE FOURTH MOMENT OF A UNIVARIATE GARCH[P,Q] PROCESS
Peter A. Zadrozny 365



CONTENTS



DEDICATION ix



LIST OF CONTRIBUTORS xi



INTRODUCTION
Thomas B. Fomby and Dek Terrell xiii



REMARKS BY ROBERT F. ENGLE III AND SIR CLIVE W. J. GRANGER, KB
Given During Third Annual Advances in Econometrics Conference at Louisiana State University, Baton Rouge, November 5-7, 2004



GOOD IDEAS
Robert F. Engle III xix



THE CREATIVITY PROCESS
Sir Clive W. J. Granger, KB xxiii



REALIZED BETA: PERSISTENCE AND PREDICTABILITY
Torben G. Andersen, Tim Bollerslev, Francis X. Diebold and Ginger Wu 1



ASYMMETRIC PREDICTIVE ABILITIES OF NONLINEAR MODELS FOR STOCK RETURNS: EVIDENCE FROM DENSITY FORECAST COMPARISON
Yong Bao and Tae-Hwy Lee 41



FLEXIBLE SEASONAL TIME SERIES MODELS
Zongwu Cai and Rong Chen 63



ESTIMATION OF LONG-MEMORY TIME SERIES MODELS: A SURVEY OF DIFFERENT LIKELIHOOD-BASED METHODS
Ngai Hang Chan and Wilfredo Palma 89



BOOSTING-BASED FRAMEWORKS IN FINANCIAL MODELING: APPLICATION TO SYMBOLIC VOLATILITY FORECASTING
Valeriy V. Gavrishchaka 123



OVERLAYING TIME SCALES IN FINANCIAL VOLATILITY DATA
Eric Hillebrand 153



EVALUATING THE 'FED MODEL' OF STOCK PRICE VALUATION: AN OUT-OF-SAMPLE FORECASTING PERSPECTIVE
Dennis W. Jansen and Zijun Wang 179



STRUCTURAL CHANGE AS AN ALTERNATIVE TO LONG MEMORY IN FINANCIAL TIME SERIES
Tze Leung Lai and Haipeng Xing 205



TIME SERIES MEAN LEVEL AND STOCHASTIC VOLATILITY MODELING BY SMOOTH TRANSITION AUTOREGRESSIONS: A BAYESIAN APPROACH
Hedibert Freitas Lopes and Esther Salazar 225



ESTIMATING TAYLOR-TYPE RULES: AN UNBALANCED REGRESSION?
Pierre L. Siklos and Mark E. Wohar 239




BAYESIAN INFERENCE ON MIXTURE-OF-EXPERTS FOR ESTIMATION OF STOCHASTIC VOLATILITY
Alejandro Villagran and Gabriel Huerta 277



A MODERN TIME SERIES ASSESSMENT OF "A STATISTICAL MODEL FOR SUNSPOT ACTIVITY" BY C. W. J. GRANGER [1957]
Gawon Yoon 297



PERSONAL COMMENTS ON YOON'S DISCUSSION OF MY 1957 PAPER
Sir Clive W. J. Granger, KB 315



A NEW CLASS OF TAIL-DEPENDENT TIME-SERIES MODELS AND ITS APPLICATIONS IN FINANCIAL TIME SERIES
Zhengjun Zhang 317


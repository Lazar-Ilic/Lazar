Preface xxvii

Microeconomics is the closest thing economics has to a canon. It comprises the essential base of knowledge for all the various forms and extensions of the discipline. Microeconomics is also extremely useful. It introduces tools that are fundamental to effective decision making in business, government, and everyday life. We believe that microeconomics should inspire and excite students with its elegance and usefulness, and that a textbook should support this goal.

Chapter 1, Adventures in Microeconomics: We open the book with a brief introductory chapter and a story about the markets for making and for buying coffee to entice and excite students about the study of microeconomics. Through an Application, a Theory and Data section, and a Freakonomics box, we show students right away how the microeconomic tools developed in this course are useful, not just in the study of economics and business, but in the pursuit of everyday life.

Chapter 2, Supply and Demand: In Chapters 2 and 3, we lay a solid foundation by going deeply into supply and demand before we move on to consumer and producer behavior. Existing books usually separate the presentation and the application of this simple yet powerful model. Presenting all aspects of this model at the beginning makes logical sense, and we (and those who class-tested the book) have experienced success with this approach in classes. Chapter 2 presents the basics of the supply-and-demand model. Of particular note is the section "Key Assumptions of the Supply and Demand Model," which exemplifies the care with which we develop and explain microeconomic theory.

Chapter 3, Using Supply and Demand to Analyze Markets: In Chapter 3, we use the supply-and-demand model to analyze extensively consumer and producer surplus, price and quantity regulations, and taxes and subsidies. We believe that the earlier these concepts are introduced and the more completely they are explained, the easier it is to use them throughout the course. Note that the presentation of the topics in Chapter 3 is designed to be flexible: You don't have to cover every topic in the chapter; you can pick and choose.

Chapter 4, Consumer Behavior: How do consumers decide what and how much to consume given the enormous variety of goods and services available to them? We begin this crucial chapter by clearly laying out, in one section, the assumptions we make about consumer behavior. Actual tests among professors consistently showed this approach as being especially helpful for their students. This chapter also introduces utility theory and consumers' budget constraints in a clear, yet rigorous, presentation.

Chapter 5, Individual and Market Demand: Here we show how consumer preferences are used to derive market demand. Section 5.3, "Decomposing Consumer Responses to Price Changes into Income and Substitution Effects," takes extra care in explaining this topic, which students often find challenging. Abundant applications and discussion of pitfalls to avoid make this material particularly accessible and interesting.

Chapter 6, Producer Behavior: How do companies decide which combination of inputs to use in production, and how does this decision affect production costs? In this chapter, we once again begin by clearly laying out the "Simplifying Assumptions about Firms' Production Behavior." Later in the chapter, we devote a complete section to the role technological change plays in firms' productivity over time. Several applications and examples (including a Freakonomics box on how cell phones have altered the behavior of producers in the Indian fish market) bring this material alive for students.

Chapter 7, Costs: Cost curves illustrate how costs change with a firm's output level and are crucial in deriving market supply. Because opportunity costs and sunk costs are often difficult concepts for students to master, we take extra care at the start of Chapter 7 to distinguish these concepts and illustrate the role they play in decision making. Our examples (including studies of gym membership usage and why movie studios release films they know will lose money) engage students so that they can better understand the often challenging concepts in this chapter.

Chapter 8, Supply in a Competitive Market: This chapter begins our coverage of market structure, and it uses real-life industries such as the Texas electricity industry and housing markets in Boston, Massachusetts, and Fargo, North Dakota, to explain how competitive markets work. We clearly, carefully, and patiently explain a firm's shutdown decision, a topic that students often find confusing.

Chapter 9, Market Power and Monopoly: This chapter begins with a thorough discussion of the origin of market power and how having such power affects a firm's production and pricing decisions. Our three-step process for determining profit maximization for a firm with market power clarifies this topic for students. We bring the concept of monopolistic market power to life using examples of real firms with nearmonopoly power, such as Durkee-Mower, Inc., the firm that makes Marshmallow Fluff, and Dr. Brown's, a manufacturer of specialty sodas. Abundant applications, including a discussion of how Southwest Airlines enters the stronghold airports of incumbent carriers and drives down fares, further engage students' interest.

Chapter 10, Market Power and Pricing Strategy: This extremely practical and useful chapter will appeal especially to business students. We thoroughly discuss the many ways in which a firm can take advantage of pricing power, and we clearly describe pricing strategies that can be effective in a variety of situations. Particularly useful to students are Figure 10.1: An Overview of Pricing Strategies and a pedagogical device called When to Use It, which explains at the start of each strategy what a firm needs to know about its market and customers to use each pricing strategy most effectively.

Chapter 11, Imperfect Competition: This chapter looks at oligopolies and monopolistically competitive firms. Unlike perfectly competitive and monopolistic firms, these firms must consider their competitors' actions and strategize to maximize their profits. To help students understand the various models of imperfect competition, each section starts with a Model Assumptions box that lists the conditions an industry must meet for that model to apply.

Chapter 12, Game Theory: The tools of game theory can be used to explain strategic interactions among firms and to predict market outcomes. Students will find our game theory analysis (presented in one chapter for better comprehension) easy to follow and understand because of our use of the check method (page 489), which simplifies games and helps students easily identify Nash equilibria and dominant/dominated strategies. Varied topics from penalty kicks in soccer to celebrity winemaking to airlines' responses to threats of entry to the movie Dr. Strangelove show the usefulness of game theory in not just business, but also everyday decision making.

Chapter 13, Investment, Time, and Insurance: Understanding the role of risk and uncertainty over time helps individuals and firms make better economic decisions about investments and insurance. We clearly explain how current costs, future payoffs, time, and uncertainty play a fundamental role in the many decisions firms and consumers face every day. Reviewers especially appreciated our coverage of all these topics in one concise chapter.

Chapter, 14 General Equilibrium: We can analyze the conditions that must hold for an economy to operate efficiently. We explain intuitively the concepts of general equilibrium, using an extension of the supply-and-demand framework. Examples include the decline in teacher quality and the interaction between housing and labor markets. We also explain the connections among exchange, input, and output efficiencies and then tie them to the Welfare Theorems.

Chapter 15, Asymmetric Information: After learning in earlier chapters what conditions must hold for markets to work well, we look at situations in which markets might not work well. Chapter 15 shows how market outcomes are distorted when information is not equally shared by all potential parties in a transaction. As always, a variety of examples from auto insurance to real estate transactions to piracy shows students that concepts learned in microeconomics are useful in many areas of life.

Chapter 16, Externalities and Public Goods: This chapter continues our examination of market failure by looking at what happens to market outcomes when transactions affect people who are neither the buying nor the selling party, and what happens when a good's benefits are shared by many people at the same time. Our coverage makes clear to students why externalities occur and how they can be remedied (including a discussion of tradable emissions permits and the Coase theorem). In our coverage of public goods, we show why a fire department might have an incentive to allow a house to burn down.

Chapter 17, Behavioral and Experimental Economics: The recent growth of behavioral economics poses a challenge to traditional microeconomics because it questions whether people actually behave the way traditional theory predicts they will. This question puts any intermediate microeconomics book in a bit of a conundrum because embracing behavioral economics seems to undermine the methods learned in the book.

Our chapter on behavioral economics explains how to think rationally in an irrational world. If some people make irrational economic decisions (and we present the behavioral evidence of situations where they tend to make mistakes), other market participants can use this irrationality to their advantage.

PART 1 BASIC CONCEPTS



1 Adventures in Microeconomics 1

Fuck yeah Lauren and Rosa fuck yeah getting high on coffee and high on life fuck yeah fuck yeah cherry Red chariot excess is just my character all Black tux[edo] ninja shoes Lavender. I never needed acceptance from all you outsiders. Climbing up the Lord's ladder... coffee beans fucking ESSKEETIT Harverd shit fucking Dr. Scott Duke Kominers and fucking Dr. Robin Hanson what do they know and Master Lazar Ilic... ESSKEETIT gang gang gang in this bitch what an adventure <3

1.1 Microeconomics [And What It Can Teach Us About Rosa And Lauren] 2

This is a nice easy EZPZ lemon squeezy written book it is very heavy handed and explanatory reading experience textual experience for the reader really gently brushes me up again recalls for me all of the fundamentals without even telling me assertively to "recall notion/concept/definition X" it just tells a nice nice nice little story about Rosa the Red and Lauren.

Learning The Tools Of Microeconomics 3

OK OK Application. Theory And Data. Freakonomics. Not a huge fan of the name but the earlier example certainly felt and seemed sharp... to help the readership become even more extra special saucy critical humans who can observe reality with the dispassionate sociopathic economist lens of putting a number on the quantitative monetary value of a human life implicit and explicitly used in the calculi etc.

Freakonomics Thomas Thwaites's Toaster 4

OK

Using The Tools Of Microeconomics 4

Yes righto how to go about technically attempting to evaluate counterfactuals and all of these "ceteris paribus" argumentations.

Make The Grade Holding The Rest Of The World Constant 5

Righto OK. What precise information we can gather and have on a topic righto right.

1.2 This Book [And How Rosa And Lauren Would See It] 5

OK

Consumers' And Producers' Decisions 6

OK

Application Better eBaying Through Microeconomics 6

Taking A Photo. Writing The Product Description. Deciding On The Auction Characteristics.

Market Supply 7

OK.

Beyond The Basics 8

Righto beyond a the basiques.

Focus On Data 9

Oh good so these George Mason University dudes might actually know something actual factual sort of kind of of like how the system operates bro the system bro the system. Bro we live in a societie bro ever fracturing lack of values bro value the values bro the bits and bytes and blip bloops in the matrices of the tensors flowing bro ESSKEETIT.

Theory And Data The Benefits Of Studying Economics 9

Oh cool there are benefits and levels to this study stuff bro lit. Oh boy philosophy majors earn only half as much as economics majors, on average bro that is a statistic.

Let The Fun Begin! 10

Ah right fun becoming an un cucked member of the George Mason University masonry of douche bros slowly but surely seeing things in even higher resolution and learning how to tool around .txt files perhaps like Dr. Scott Duke Kominers might know how to try and rouse people into Leftism on some positions and issues with bourgeois economiques theorie.

Summary 10



Review Questions 11



2 Supply And Demand 13

World Of Warcraft and virtual Gold bro dude real Gold bro ESSKEETIT Gold Medal and some real money bro assets bro Gold metal bro. Gold rush bro.

2.1 Markets And Models 14

OK.

What Is A Market? 14



Key Assumptions Of The Supply And Demand Model 15

We Restrict Our Focus to Supply And Demand In A Single Market. All Goods Bought And Sold In The Market Are Identical. All Goods Sold In The Market Sell For The Same Price And Everyone Has The Same Information About Prices, The Quality Of The Goods Being Sold, And So On. There Are Many Buyers And Sellers In The Market.

2.2 Demand 16

Pike Place Market, one of the best known public markets in the world, spans several blocks in the northwest corner of downtown Seattle. It has operated continually since 1907, and on any given day hosts hundreds of vendors selling everything from fish and meat to produce and flowers to crafts and antiques. The market sees approximately 10 million visitors per year.

I done went there a long long long time ago that was not lifetimes ago but it was a long good while back to be sure bro.

Factors That Influence Demand 16

Price. The Number Of Consumers. Consumer Income Or Wealth. Consumer Tastes. Prices Of Other Goods.

Demand Curves 17

Graphical Representation Of The Demand Curve. Demand For Tomatoes wow people do actually demand this stuff... I forget how nice some cooked warm soft wummy tummy tomatoes can be and have some unironic Magnesium content too on top of a bunch of other wummies. Boil boil pasta pot. Mathematical Representation Of The Demand Curve.

Shifts In Demand Curves 19

Shifts In The Demand Curve. Prevalence Of Smoking By Education Category In The United States, Age 25 And Older, 1940-2000.

Theory And Data Changes In Taste And The Demand For Cigarettes 20



Why Is Price Treated Differently From The Other Factors That Affect Demand? 22

There are at least three reasons why economists focus on the effects of a change in a good's price. First, price is typically one of the most important factors that influence demand. Second, prices can usually be changed frequently and easily. Therefore, when we contemplate how markets respond to changes or "shocks," price fluctuations are likely to be a common response. Third, and most important, of all the factors that influence demand, price is the only one that also exerts a large, direct influence on the other side of the market - on the quantity of the good that producers are willing to supply. Price therefore serves as the critical element that ties together demand and supply. Let's turn to that other side of the market now.

2.3 Supply 22

We have one half of the demand and supply model assembled. In this section, we present the supply half. By supply we mean the combined amount of a good that all producers in a market are willing to sell.

Factors That Influence Supply 22

Price. Suppliers' Costs Of Production. The Number Of Sellers. Sellers' Outside Options.

Supply Curves 23

Graphical Representation Of The Supply Curve. Mathematical Representation Of The Supply Curve.

Inverse Supply Curves. Supply Choke Price.

Shifts In The Supply Curve 24

Changes In Quantity Supplied. Shifts In The Supply Curve. Change In Supply.

Why Is Price Also Treated Differently For Supply? 25



2.4 Market Equilibrium 25

Market Equilibrium. Equilibrium Price.

The Mathematics Of Equilibrium 26



Make The Grade Does Quantity Supplied Equal Quantity Demanded In Equilibrium? 26



Why Markets Move Toward Equilibrium 27

Excess Supply. Excess Demand. Adjusting To Equilibrium.

Figure It Out 2.1 28



The Effects Of Demand Shifts 29

Shifts In Curves Versus Movement Along A Curve.

Figure It Out 2.2 30



Freakonomics The Price Of Fame: President Obama And The Paparazzi 31



The Effects Of Supply Shifts 32



Figure It Out 2.3 33



Summary Of Effects 34

Interesting.

Application Supply Shifts And The Video Game Crash Of 1983 35



Figure It Out 2.4 36

Well we sure are figuring something out for sure figuring out how to gently recall all of the old basics of reality actually! Wowzerz!

Make The Grade Did The Curve Shift, Or Was It Just A Movement Along The Curve? 37

Ah yes this was a recurring theme of simple manipulations in my University Of Texas At Austin economics courses.

What Determines The Size Of Price And Quantity Changes? 38

Size Of The Shift. Size Of Equilibrium Price And Quantity Changes, And The Slopes Of The Supply And Demand Curves. Slopes Of The Curves.

Application The Supply Curve Of Housing And Housing Prices: A Tale Of Two Cities 39

Population Indices For New York And Houston, 1977-2009. Housing Price Indices For New York And Houston, 1977-2009.

Changes In Market Equilibrium When Both Curves Shift 41

Example Of A Simultaneous Shift In Demand And Supply. When Both Curves Shift, The Direction Of Either Price Or Quantity Will Be Ambiguous.

2.5 Elasticity 42

Elasticity. Price Elasticity Of Demand. Ah yes more of the basics.

Slope And Elasticity Are Not The Same 43



The Price Elasticities Of Demand And Supply 43



Price Elasticities And Price Responsiveness 44



Application Demand Elasticities And The Availability Of Substitutes 45

Elasticities And Time Horizons. Terms For Elasticities By Magnitude. Elastic. Inelastic. Unit Elastic. Perfectly Inelastic. Perfectly Elastic.

Elasticities And Linear Demand And Supply Curves 46

Elasticity Of A Linear Demand Curve. Elasticity Of A Linear Supply Curve. Elasticity Of A Linear Supply Curve.

Figure It Out 2.5 49



Perfectly Inelastic And Perfectly Elastic Demand And Supply 49

Perfect Inelasticity. Perfect Elasticity. Perfectly Inelastic And Perfectly Elastic Demand Curves.

The Price Elasticity Of Demand, Expenditures, And Revenue 51

Expenditures Along A Linear Demand Curve.

Income Elasticity Of Demand 53

Income Elasticity Of Demand. Inferior Goods. Normal Goods. Luxury Goods.

Cross-Price Elasticity Of Demand 53

Cross-Price Elasticity Of Demand.

Figure It Out 2.6 54



2.6 Conclusion 55



Summary 55



Review Questions 56



Problems 57



3 Using Supply And Demand To Analyse Markets 61



3.1 Consumer And Producer Surplus: Who Benefits In A Market? 62

In Chapter 2, we introduced the tools of supply and demand analysis. We learned about the economic decisions that supply and demand curves embody, and defined what it means for a market to be in equilibrium. In this chapter, we put those tools to work to take a deeper look at how markets operate. We study how to measure the total benefits that consumers and producers gain in any given market, and how these benefits change when demand or supply shifts. We also see how various government interventions into markets affect the well-being of consumers and producers.

Consumer Surplus 62

Consumer Surplus. Demand Choke Price.

Producer Surplus 63

Producer Surplus. Defining Consumer Surplus. Defining Producer Surplus.

Figure It Out 3.1 65



Application The Value Of Innovation 66

Ah ha yes finally the value... of innovation was it oh yeah new ideas great ideas the best of ideas it was the worst of ideas... ideation and innovation ought to be rewarded in school and hopefully by the markets... that example of getting paid more than you are worth so that firing is a punishment was kind of interesting... err non sequitur like I mean if you can just get another dude to pay you the same more than you worth then it does not really work out that way the "logic" of the argument was not quite all that clear to me. In any case of course we were told in high schools that entrepreneurs who got capital were litty lit lit so there you go apply to Y Combinator 69 times bro OXFERD ESSKEETIT.

Consumer Surplus And The Elasticity Of Demand.

Application What Is LASIK Eye Surgery Worth To Patients? 68

This was a fascinating product yeah yeah I remember when this shit was like new and Westlakers a pop off for this stuff bro they would advertise this stuff on the radio like and people a do it. For thous bro Gs stacks bro. Anyways I think laser hair removal may be OK for my skin organ do it sometime bro go through the sessions see how I am feeling maybe just try a small patch and see how that goes what I think.

Valuing LASIK Eye Surgery

The Distribution Of Gains And Losses From Changes In Market Conditions 69

Changes In Surplus From A Supply Shift.

Application How Much Did 9/11 Hurt The Airline Industry? 70

Airlines And September 11

Figure It Out 3.2 73



3.2 Price Regulations 75

Politicians call regularly for price ceilings on products whose prices have risen a lot. In this section, we explore the effects of direct government interventions in market pricing. We look both at regulations that set maximum prices [like a gas price ceiling] and minimum prices [price floors like a minimum wage].

Price Ceilings 75

Price Ceiling. The Effects Of A Price Ceiling. Graphical Analysis. Transfer. Deadweight Loss. Analysis Using Equations. The Problem Of Deadweight Loss. Importance Of Price Elasticities. Nonbinding Price Ceilings. Deadweight Loss And Elasticities.

Price Floors 81

Price Floor. Price Support. The Effects Of A Price Floor. Nonbinding Price Floors.

3.3 Quantity Regulations 83

Sometimes, rather than regulating prices, governments impose quantity regulations. We discuss some of these regulations and analyze their effects on market outcomes in this section.

Quotas 83

Quota. Graphical Analysis. Analysis Using Equations. The Effects Of A Quota.

Government Provision Of Goods And Services 86

The Effects Of Government Provision Of Education.

Theory And Data Does Public Health Insurance Crowd Out Private Insurance? 88

Interesting topic to be sure I know the standard Leftist narrative is that we want public health insurance for all Amerikans.

Government Provision Of Health Insurance.

3.4 Taxes 90



Tax Effects On Markets 90

Effect Of A Tax On Boston Movie Tickets. Graphical Analysis. Analysis Using Equations.

Why Taxes Create A Deadweight Loss 94

OK this is still all basically good strong solid review of the University Of Texas At Austin economics courses which I did take.

Why A Big Tax Is Much Worse Than A Small Tax 95

Obvious. The Effect Of A Larger Tax On Boston Movie Tickets.

The Incidence Of Taxation: The Payer Doesn't Matter 96

Tax Incidence.

Make The Grade Did I Measure Deadweight Loss Correctly? 97

Elastic Demand With Inelastic Supply. Inelastic Demand With Elastic Supply. Tax Incidence And Elasticities.

Figure It Out 3.3 99



3.5 Subsidies 100

Subsidy.

The Impact Of A Producer Subsidy

Application The Cost Of The Black-Liquor Loophole 102

Fascinating that was a loop hole on fuel usage. Fascinating stuff so maybe I could simply try and get employed at one of these big firms as some kind of a dude a legal trawler who finds economically useful such loop holes in the laws and then that firm can invoke things and make more money!

Figure It Out 3.4 103



Freakonomics Can Economic Incentives Get You Pregnant? 105

Yeah this is very critical and important for sure to Red Pill the readership on how like Behavioural Economiques is real bro the gubberment controls and trolls you bro incentives bro the strooture of reality bro OXFERD ESSKEETIT. Ceausescu was he a brutal monster dictator or like a the good guy? Some dude for sure. For sure know a lot of Romanian dudes actually learned French moved to France good for them.

3.6 Conclusion 106



Summary 106



Review Questions 107



Problems 107



PART 2 CONSUMPTION AND PRODUCTION



4 Consumer Behavior 111

Ah righto right the behaviour of consumers... well we do know that many are seriously under informed, ignorant, they actually do pay attention to catastrophically low information advertising and propaganda... many do not actually try and test out alternatives in order to fully optimise... like a serious life style web logger ought to at the very least so that others can benefit from their useful discoveries.

I just forget sometimes that there are people out there who do not use Library Genesis or torrent cinema films so...

These are OK examples.

4.1 The Consumer's Preferences And The Concept Of Utility 112

Consumers' preferences underlie every decision they make. Economists think of consumers as making rational choices about what they like best, given the constraints that they face when they make their choices.

Assumptions About Consumer Preferences 113

Completeness And Rankability. Consumption Bundle. For Most Goods, More Is Better Than Less [Or At Least More Is No Worse Than Less]. Transitivity. The More A Consumer Has Of A Particular Good, The Less She Is Willing To Give Up Of Something Else To Get Even More Of That Good.

The Concept Of Utility 114

Utility. Utility Function.

Marginal Utility 114

Marginal Utility.

Utility And Comparisons 115

Interpseronal Comparisons. Welfare Economics.

4.2 Indifference Curves 116

Indifferent. Indifference Curve. Building An Indifference Curve. A Consumer's Indifference Curves.

Characteristics Of Indifference Curves 118

OK righto I mean... still thinking about how like the more you consume classical musique the higher the marginal utility goes. Indifference Curves Cannot Cross. Tradeoffs Along An Indifference Curve.

Make The Grade Draw Some Indifference Curves To Really Understand The Concept 119



The Marginal Rate Of Substitution 120

The Slope Of An Indifference Curve Is The Marginal Rate Of Substitution
Marginal Rate Of Substitution Of X For Y.

Freakonomics Do Minnesotans Bleed Purple? 122



The Marginal Rate Of Substitution And Marginal Utility 123



The Steepness Of Indifference Curves 124



Figure It Out 4.1 125



Theory And Data Indifference Curves Of Phone Service Buyers 126



The Curvature Of Indifference Curves: Substitutes And Complements 127

OK

Figure It Out 4.2 131

OK

Application Indifference Curves For "Bads" 132



4.3 The Consumer's Income And The Budget Constraint 133



The Slope Of The Budget Constraint 135



Factors That Affect The Budget Constraint's Position 135



Figure It Out 4.3 137



Nonstandard Budget Constraints 138



4.4 Combining Utility, Income, And Prices: What Will The Consumer Consume? 140



Solving The Consumer's Optimisation Problem 140



Figure It Out 4.4 143



Implications Of Utility Maximisation 143

Two Consumers' Optimal Choices.

Theory And Data Indifference Curves Of Phone Service Buyers Revisited 145



A Special Case: Corner Solutions 147

Corner Solution. Interior Solution.

Figure It Out 4.5 148



An Alternative Approach To Solving The Consumer's Problem: Expenditure Minimisation 149

Constrained Maximisation Problem. Utility Maximisation Versus Expenditure Minimisation.

4.5 Conclusion 151

Righto fascinating stuff concepts notions implicit in lame human agents boundedly rational and so on and so on.

This chapter has looked at how consumers decide what to consume. This decision combines two characteristics of consumers, their preferences [embodied in their utility function] and their income, and one characteristic of the market, the goods' prices.

We saw that a consumer will maximise her utility from consumption when she chooses a bundle of goods such that the marginal rate of substitution between the goods equals their relative prices. That is, in this bundle the ratio of the goods' utilities equals their price ratio. Equivalently, the goods' marginal utilities per dollar spent are equal. If this property didn't hold, a consumer could make herself better off by consuming more of the goods with high marginal utilities per dollar and less of the goods with low marginal utilities per dollar.

There is another way to think about the consumer's problem of what and how much to consume. Rather than thinking of consumers as trying to maximise utility subject to a budget constraint, we could think of them as trying to minimise the expenditure necessary for them to reach a given level of utility. This is called the expenditure minimisation problem. We saw how it turns out that this delivers the same rule for optimal consumption behavior: The MRS of the goods should equal their price ratio.

Summary 151

Utility is the economic concept of consumers' happiness or well-being, and the utility function is the construct that relates the amount of goods consumed (the inputs) to the consumer's utility level (the output). There are properties that we expect almost all utility functions to share: the completeness, rankability, and transitivity of utility bundles, that having more of a good is better than having less, and that the more a consumer has of a particular good, the less willing she is to give up something else to get more of that good. [Section 4.1]

Consumers' preferences are reflected in their indifference curves, which show all the combinations of goods over which a consumer receives equal utility. The set of properties imposed on utility functions imply some restrictions on the shapes of indifference curves. Namely, indifference curves slope downward, never cross for a given individual, and are convex to the origin. [Section 4.2]

The negative of the slope of the indifference curve is the marginal rate of substitution of good X for good Y (MR S XY). The MRS is the ratio of the marginal utilities of the goods in the utility function. [Section 4.2]

Consumer preferences lead to differences in the steepness and curvature of indifference curves. If a consumer views two goods as perfect substitutes or perfect complements, their indifference curves will be shaped like straight lines and right angles, respectively. [Section 4.2]

The consumer's decision about how much of each good to consume depends not only on utility, but also on how much money that person has to spend (her income) and on the prices of the goods. In analyzing the role of income in consumption decisions, we assume the following: Each good has a fixed price, and any consumer can buy as much of a good as she wants at that price if the consumer has sufficient income to pay for it; the consumer has some fixed amount of income to spend; and the consumer cannot save or borrow.

The budget constraint captures both a consumer's income and the relative prices of goods. The constraint shows which consumption bundles are feasible (i.e., affordable given the consumer's income) and which are infeasible. The slope of the budget constraint is the negative of the ratio of the prices of the two goods ( P X / P Y). [Section 4.3]

The consumer's decision is a constrained-optimization problem: to maximize utility while staying within her budget constraint. The utility-maximizing solution is generally to consume the bundle of goods located where an indifference curve is tangent to the budget constraint. At this optimal point, the consumer's marginal rate of substitution - the ratio of the consumer's marginal utilities from the goods - equals the goods' relative price ratio.

A corner solution, where the optimal quantity consumed of one good is zero, can occur when a consumer's marginal utility of a good is so low compared to that good's relative price that she is better off not consuming any of that good at all. In such cases, the MRS does not equal the price ratio even though the consumer is at the utility-maximising consumption bundle. [Section 4.4]

The consumer's problem of what and how much to consume can be recast as an expenditure-minimisation problem. That is, rather than thinking of consumers as trying to maximise utility subject to a budget constraint, we could think of them as trying to minimize the expenditure necessary for them to reach a given level of utility. The optimal choices for both problems result in the same criterion: The MRS of the goods should equal their price ratio. [Section 4.4]



Review Questions 152



Problems 152



Appendix The Calculus Of Utility Maximisation And Expenditure Minimisation 157

Righto the invisible cloak hand.

Ah now this is actually a good section which goes into basic calculus now I think this book is fairly simple and weak but this goes beyond I should go to a more advanced textbook after this one.

OK OK microeconomics session good little brush up.

Figure It Out 4A.1 161



5 Individual And Market Demand 165



5.1 How Income Changes Affect An Individual's Consumption Choices 166

OK

Normal And Inferior Goods 167



Income Elasticities And Types Of Goods 168



The Income Expansion Path 169



The Engel Curve 170



Application Engel Curves And House Sizes 171

OK

Figure It Out 5.1 173

OK

5.2 How Price Changes Affect Consumption Choices 174



Deriving A Demand Curve 174



Shifts In The Demand Curve 176

OK

Freakonomics Even Animals Like Sales 177



Figure It Out 5.2 178



5.3 Decomposing Consumer Responses To Price Changes Into Income And Substitution

OK

Effects 180



Isolating The Substitution Effect 181

OKOK

Isolating The Income Effect 183



Make The Grade Computing Substitution And Income Effects From A Price Change 183

OK

The Total Effects 184



What Determines the Size of the Substitution and Income Effects? 184



Figure It Out 5.3 186



Application Backward-Bending Labor Supply And Income Effects In Leisure 187



Theory And Data Golfers' Backward-Bending Labor Supply Curves 189

OK

An Example of the Income Effect with an Inferior Good 190

OK

Giffen Goods 192

OK

Application In Search of Giffen Goods 194



Make the Grade Simple Rules to Remember about Income and Substitution Effects 195

OK

5.4 The Impact of Changes in Another Good's Price: Substitutes and Complements 195



A Change in the Price of a Substitute Good 196



Indifference Curve Shapes, Revisited 198



Application Movies in a Theater and at Home-Substitutes or Complements? 198

OK

5.5 Combining Individual Demand Curves to Obtain the Market Demand Curve 199

OK

The Market Demand Curve 201

OK

Using Algebra to Move from Individual to Market Demand 201



Make the Grade Adding Demand Curves Horizontally, Not Vertically 202

OK

Figure It Out 5.4 202



5.6 Conclusion 203

OKOK

Summary 203



Review Questions 204

OK

Problems 204

OK

Appendix The Calculus of Income and Substitution Effects 209



Figure It Out 5A.1 211



6 Producer Behavior 215



6.1 The Basics of Production 216



Simplifying Assumptions about Firms' Production Behavior 216

OK

Application Do Firms Really Always Minimize Costs? 218



Production Functions 219



6.2 Production in the Short Run 220

OK

Marginal Product 220

OK

Average Product 223

OK

Figure It Out 6.1 223

OK

Application How Short Is the Short Run? 224



6.3 Production in the Long Run 224

OK

The Long-Run Production Function 225

OK

6.4 The Firm's Cost-Minimization Problem 225



Isoquants 226

OK

Isocost Lines 229

OK

Figure It Out 6.2 232



Identifying Minimum Cost: Combining Isoquants And Isocost Lines 233

OK

Cost Minimization—A Graphical Approach 233



Input Price Changes 235

OK

Figure It Out 6.3 235



Theory and Data Hospitals' Input Choices And Medicare Reimbursement Rules 237

OK

6.5 Returns To Scale 238

OK

Factors Affecting Returns To Scale 238

OK

Make The Grade How To Determine A Production Function's Returns To Scale 240



Figure It Out 6.4 240

OK

6.6 Technological Change 241

OK yeah bro me and Elon Musk hop in the mix just ad hoc de novo Sherlock forces the fucking eschaton bro and light it up Fully Automated Luxury Communism gang bro take that ya dumb fucking Tyler Cowen clownerino happy dudes over there straight chilling on this fine afternoon in George Mason University masonry. Alright alright no more "politics" trolling in my serious serious serious mid brow Notes.

Application Technological Change in U.S. Manufacturing 242



Freakonomics Why Do Indian Fishermen Love Their Cell Phones So Much? 244



6.7 The Firm's Expansion Path and Total Cost Curve 246

OK OK

6.8 Conclusion 247

OKOK

Summary 247

OK

Review Questions 248



Problems 248

OK

Appendix The Calculus Of Cost Minimisation 253

OK

Figure It Out 6A.1 257

OK

Figure It Out 6A.2 259

OK

7 Costs 261

OK

7.1 Costs That Matter for Decision Making: Opportunity Costs 262



Application Making Big Money by Not Operating Your Business - A Lesson about

OK

Opportunity Cost 263

OK

Figure It Out 7.1 264

OK

Freakonomics Paying for Papers: The Economics of Cheating 264

Righto because the internet.

7.2 Costs That Do Not Matter for Decision Making: Sunk Costs 265

OK

Sunk Costs and Decisions 266

OKOK

Theory and Data Gym Memberships 268

OK right.

Application Why Do Film Studios Make Movies That They Know Will Lose Money? 268

OK

7.3 Costs and Cost Curves 270

OK OK

Fixed Cost 270

OK

Variable Cost 270

OK

Flexibility and Fixed versus Variable Costs 271

OK

Deriving Cost Curves 272



Fixed Cost Curve 274

OKOK

Variable Cost Curve 274

OK

Total Cost Curve 274



7.4 Average and Marginal Costs 274

OK

Average Cost Measures 274

OK

Marginal Cost 276

OK

Figure It Out 7.2 277

OK

Relationships Between Average And Marginal Costs 278

OK

Figure It Out 7.3 280

OK

7.5 Short-Run And Long-Run Cost Curves 281

OK

Short-Run Production And Total Cost Curves 281

OK

Short-Run Versus Long-Run Average Total Cost Curves 283

OK

Figure It Out 7.4 285



Short-Run Versus Long-Run Marginal Cost Curves 286

OK

7.6 Economies In The Production Process 287



Economies Of Scale 287

OK

Economies Of Scale Versus Returns To Scale 287

OK

Figure It Out 7.5 288

OK OK

Application Economies Of Scale And Makin' Bacon 288

OK

Economies Of Scope 290

OK OK

Why Economies Of Scope Arise 291

OK OK

7.7 Conclusion 291

OK this was in fact a super solid chapter I like this book it is a nice gentle super duper maybe this was for the Harverd non advanced course but it is super soft gentle EZPZ lemon squeezy breezy reading I like it I dig nice pleasant book who knows we see I really need to get higher brow for a firm or not really I mean this is pretty OK to see a few hundred little shticks and examples see a few hundred more won't hurt me. Then another few hundred stack of a trillion examples really take the micro economiques pills.

Summary 291

OK

Review Questions 292

OK

Problems 292

OK OK OK

Appendix The Calculus Of A Firm's Cost Structure 297

OK

Figure It Out 7A.1 299

OK OK OK OK OK

PART 3 MARKETS AND PRICES



8 Supply in a Competitive Market 303

OK

8.1 Market Structures and Perfect Competition in the Short Run 304

OK eggs eggy weggies yeah yeah.

Perfect Competition 305

OK OK

The Demand Curve as Seen by a Price Taker 306

OK

8.2 Profit Maximisation in a Perfectly Competitive Market 306

OK

Total Revenue, Total Cost, and Profi t Maximization 307

OK

How a Perfectly Competitive Firm Maximizes Profit 308

OK

Application Do Firms Always Maximize Profits? 310

OK

Measuring a Firm's Profit 311

OK

Figure It Out 8.1 312

OK

Figure It Out 8.2 313

OK

Make the Grade A Tale of Three Curves 314



8.3 Perfect Competition in the Short Run 315

OK

A Firm's Short-Run Supply Curve in a Perfectly Competitive Market 315

OK

Application The Supply Curve of a Power Plant 316

OK

The Short-Run Supply Curve for a Perfectly Competitive Industry 317

OK

Application The Short-Run Supply of Crude Oil 319



Producer Surplus for a Competitive Firm 320

OK

Producer Surplus and Profit 321

OK

Producer Surplus for a Competitive Industry 322

OK

Figure It Out 8.3 322

OK

Application Short-Run Industry Supply And Producer Surplus In Electricity



Generation 323

OK

8.4 Perfectly Competitive Industries in the Long Run 325

OK

Entry 326



Exit 327

OK

Graphing the Industry Long-Run Supply Curve 328

OK

Theory And Data Entry And Exit At Work In Markets-Residential Real Estate 329

OK

Freakonomics The Not-So-Simple Economics Of Blackmail 330



Adjustments Between Long-Run Equilibria 331



Application The Increased Demand For Corn 333

OK

Figure It Out 8.4 334

OK

Long-Run Supply in Constant-, Increasing-, and Decreasing-Cost Industries 335

OK

8.5 Producer Surplus, Economic Rents, and Economic Profits 336

OK

Cost Differences and Economic Rent in Perfect Competition 336

OK

8.6 Conclusion 338

OK

Summary 338

OK

Review Questions 339

OK

Problems 339

OK

9 Market Power and Monopoly 347

OK

9.1 Sources of Market Power 348

OK

Extreme Scale Economies: Natural Monopoly 348

OK

Application Natural Monopoly in Satellite Radio 349

OK

Switching Costs 350

OK

Product Differentiation 351

OK

Absolute Cost Advantages or Control of Key Inputs 351

OK

Application Controlling a Key Input—The Troubled History of Fordlandia, Brazil 351

OK

Government Regulation 352

OK

Where There's a Will [And Producer Surplus], There's a Way 352

OK

9.2 Market Power and Marginal Revenue 352

OK

Freakonomics Why Drug Dealers Want Peace, Not War 353

OK

Market Power and Monopoly 354

OK

Marginal Revenue 355

OK

Figure It Out 9.1 359

OK

9.3 Profit Maximisation For A Firm With Market Power 360

OK

How to Maximise Profit 360

OK

Profit Maximisation with Market Power: A Graphical Approach 361

OK

Profit Maximisation with Market Power: A Mathematical Approach 362

OK

Figure It Out 9.2 363

OK

A Markup Formula for Companies with Market Power: The Lerner Index 363

OK

Application Market Power versus Market Share 366

OK

The Supply Relationship for a Firm with Market Power 366

OK

9.4 How a Firm with Market Power Reacts to Market Changes 367

OK

Response to a Change in Marginal Cost 367

OK

Response to a Change in Demand 368

OK

The Big Difference: Changing the Price Sensitivity of Customers 369

OK

Figure It Out 9.3 370

OK

9.5 The Winners and Losers from Market Power 371

OK

Consumer and Producer Surplus under Market Power 371



Consumer and Producer Surplus under Perfect Competition 372

OK

Application Southwest Airlines 373

OK

The Deadweight Loss of Market Power 373

OK

Differences in Producer Surplus for Different Firms 374

OK

Figure It Out 9.4 375



9.6 Governments and Market Power: Regulation, Antitrust, and Innovation 376

OK

Direct Price Regulation 376

OK

Antitrust 377



Promoting Monopoly: Patents, Licenses, and Copyrights 378



Application Internet File-Sharing and the Music Industry 379

OK interesting one now that Qobuz has the 24 44.1 and 24 192 .FLAC market on lock tho for the most part.

Patent Protection in Practice 380



Theory and Data Determining a New Drug's Potential Market 381

OK interesting that the logo reminds me of the Codeforces logo.

9.7 Conclusion 382

OK

Summary 383

OK

Review Questions 383



Problems 383

OK

Appendix The Calculus of Profit Maximization 389



Figure It Out 9A.1 391

OK

Figure It Out 9A.2 392

OK

10 Market Power and Pricing Strategies 395



10.1 The Basics of Pricing Strategy 396

OK

When Can a Firm Pursue a Pricing Strategy? 396

OK

10.2 Direct Price Discrimination I: Perfect/First-Degree Price Discrimination 398



When to Use It Perfect/First-Degree Price Discrimination 398

OK

Figure It Out 10.1 400

OK

Examples of Perfect Price Discrimination 402

OK

Application How Priceline Learned That You Can't Price-Discriminate without Market

OK

Power 402

OK

10.3 Direct Price Discrimination II: Segmenting/Third-Degree Price Discrimination 403

OK

When to Use It Segmenting/Third-Degree Price Discrimination 403

OK

The Benefits of Segmenting: A Graphical Approach 404

OK

The Benefits of Segmenting: A Mathematical Approach 406

OK

How Much Should Each Segment Be Charged? 408

OK

Make the Grade Is It Really Price Discrimination? 410

OK

Figure It Out 10.2 410



Ways to Directly Segment Customers 411

OK

Freakonomics Victoria's Not-So-Secret Price Discrimination 411

Ooh lala back to the ol' story of the Victoria's Secret firm and I am always dying to know all of Victoria's secrets you know Victoria Xia! Apply to that firm too! Hot hot hot hot!

Theory and Data Segmenting by Location in the European Market for Cars 413

OK

10.4 Indirect/Second-Degree Price Discrimination 414

OK

When to Use It Indirect/Second-Degree Price Discrimination 414

OK

Indirect Price Discrimination through Quantity Discounts 414



Figure It Out 10.3 418

OK

Indirect Price Discrimination through Versioning 419

OK

Indirect Price Discrimination through Coupons 421

OK

10.5 Bundling 422

OK

When to Use It Bundling 422

OK

Mixed Bundling 424

OK

Figure It Out 10.4 425



10.6 Advanced Pricing Strategies 427

OK

When to Use It Block Pricing and Two-Part Tariffs 427

OK

Block Pricing 428



Two-Part Tariffs 429



Figure It Out 10.5 430

OK

10.7 Conclusion 432

OK

Summary 433

OK

Review Questions 433

OK

Problems 434

OK

11 Imperfect Competition 439

OK

11.1 What Does Equilibrium Mean in an Oligopoly? 440



Application An Example of Nash Equilibrium: Marketing Movies 441

OK

11.2 Oligopoly with Identical Goods: Collusion and Cartels 442

OK

Model Assumptions Collusion and Cartels 442



The Instability of Collusion and Cartels 443

OK

Application OPEC and the Control of Oil 445



Figure It Out 11.1 447

OK

What Makes Collusion Easier? 447



Application The Indianapolis Concrete Cartel 448

OK

Freakonomics How the Government Lost the Fight against Big Tobacco 449



11.3 Oligopoly with Identical Goods: Bertrand Competition 450



Model Assumptions Bertrand Competition with Identical Goods 450

OK

Setting Up the Bertrand Model 450



Nash Equilibrium of a Bertrand Oligopoly 451



Theory and Data Computer Parts I 452

OK

11.4 Oligopoly with Identical Goods: Cournot Competition 453



Model Assumptions Cournot Competition with Identical Goods 453



Setting Up the Cournot Model 453

OK

Equilibrium in a Cournot Oligopoly 454

OK

Figure It Out 11.2 458

OK

Comparing Cournot to Collusion and to Bertrand Oligopoly 459



What Happens If There Are More Than Two Firms in a Cournot Oligopoly? 460



Cournot versus Bertrand: Extensions 460

OK

11.5 Oligopoly with Identical Goods: Stackelberg Competition 461



Model Assumptions Stackelberg Competition with Identical Goods 461



Stackelberg Competition and the First-Mover Advantage 462

OK

Figure It Out 11.3 464



11.6 Oligopoly with Diff erentiated Goods: Bertrand Competition 465



Model Assumptions Bertrand Competition with Diff erentiated Goods 465

OK

Equilibrium in a Diff erentiated-Products Bertrand Market 466



Figure It Out 11.4 469



Theory and Data Computer Parts II—Differentiation Out of Desperation 470



11.7 Monopolistic Competition 471



Model Assumptions Monopolistic Competition 471



Equilibrium in Monopolistically Competitive Markets 472

OK

Figure It Out 11.5 475

OK

11.8 Conclusion 475

OK

Summary 476

OK

Review Questions 476



Problems 477

OK

12 Game Theory 483

OK

12.1 What Is a Game? 485

OK

Dominant and Dominated Strategies 485

OK

12.2 Nash Equilibrium in One-Period Games 487

OK

Make the Grade The Check Method 489



Figure It Out 12.1 490

OK

Multiple Equilibria 491



Mixed Strategies 493



Application Random Mixed Strategies in Soccer 494

OK

The Maximin Strategy [Or: What If My Opponent Is an Idiot?] 495

OK

Application Fun in the Sun: Wine Making for Irrational Billionaires 496

OK

12.3 Repeated Games 497

OK

Finitely Repeated Games 498

OK

Infinitely Repeated Games 499

OK

Figure It Out 12.2 501

OK

12.4 Sequential Games 503



Make the Grade Backward Induction and Trimming Trees 505

OK

Another Sequential Game 506

OK

Figure It Out 12.3 507



12.5 Strategic Moves, Credibility, and Commitment 509

OK

Side Payments 510



Commitment 510

OK

Figure It Out 12.4 512

OK

Application Dr. Strangelove and the Perils of Secrecy 513

OK

Entry Deterrence: Credibility Applied 513



Theory and Data Incumbent Airlines' Responses to Threatened Entry by Southwest 515

OK

Reputation 517



Freakonomics How Game Theory Just Might Save Your Life 518

OK

12.6 Conclusion 518

OK

Summary 519

OK

Review Questions 519

OK

Problems 520

OK

Part 4 Beyond The Basics

OK

13 Investment, Time, And Insurance 527

OK

13.1 Present Discounted Values Analysis 528

OK

Interest Rates 529



The "Rule of 72" 530

OK

Present Discounted Value 530



Application The Present Discounted Value of Bonds 534



Figure It Out 13.1 536

OK

13.2 Evaluating Investment Choices 536

OK

Net Present Value 536



The Key Role of Interest Rates in Determining NPV 538

OK

Application Replacing Planes 539



NPVs versus Payback Periods 541



Figure It Out 13.2 541



13.3 The Correct Interest Rate to Use, and Capital Markets 542

OK

Nominal versus Real Interest Rates 542



Other Rate Adjustments 542



Capital Markets and the Determination of the Market Interest Rate 543

OK

Figure It Out 13.3 544

OK

13.4 Evaluating Risky Investments 545



NPV with Uncertainty: Expected Value 545



Risk and the Option Value of Waiting 546



Figure It Out 13.4 548

OK

13.5 Uncertainty, Risk, and Insurance 549

OK

Expected Income, Expected Utility, and the Risk Premium 549



Insurance Markets 551

OK yeah yeah.

Theory and Data The Insurance Value of Medicare 553



The Degree of Risk Aversion 554

OK

Risk Aversion and Investment Decisions 555



Freakonomics Apocalypse Now: How Much Will We Pay to Prevent Disaster? 555

OK

Figure It Out 13.5 556



13.6 Conclusion 557



Summary 558



Review Questions 558



Problems 558

OK

14 General Equilibrium 563



14.1 General Equilibrium Effects in Action 564

OK

An Overview of General Equilibrium Effects 565



Quantitative General Equilibrium: The Corn Example with Demand-Side Market Links 568

OK

Figure It Out 14.1 571



Theory and Data The General Equilibrium of Carmageddon 572



Quantitative General Equilibrium: The Corn Example with Supply-Side Market Links 572

OK

Freakonomics Where Have All the Good Teachers Gone? 575



Application General Equilibrium Interaction of Cities' Housing and Labor Markets 576



14.2 General Equilibrium: Equity and Efficiency 577

OK

Standards for Measuring Market Performance: Social Welfare Functions 578



Figure It Out 14.2 579

OK

Standards for Measuring Market Performance: Pareto Efficiency 580

OK

Looking for Pareto Efficiency In Markets 580



Efficiency in Markets-Three Requirements 581

OK

14.3 Efficiency In Markets: Exchange Efficiency 581



The Edgeworth Box 582

OK

Gains from Trade in the Edgeworth Box 583



Figure It Out 14.3 587

OK

14.4 Efficiency in Markets: Input Efficiency 588



The Production Possibilities Frontier 590



Figure It Out 14.4 592

OK

14.5 Efficiency in Markets: Output Efficiency 592



The Marginal Rate of Transformation 592



Figure It Out 14.5 595

OK

14.6 Markets, Efficiency, and the Welfare Theorems 596



Application Output Efficiency among Manufacturing Firms in India 597



14.7 Conclusion 599

OK

Summary 599

OK

Review Questions 600



Problems 600

OK

15 Asymmetric Information 605

OK

15.1 The Lemons Problem and Adverse Selection 606



Observable Quality 606

OK

Unobservable Quality 607

OK

Adverse Selection 607



Other Examples of the Lemons Problem 609

OK

Institutions That Mitigate Lemons Problems 609

OK

Figure It Out 15.1 612



Application Reputations in Collectibles Sales 612

OK

Adverse Selection When the Buyer Has More Information: Insurance Markets 614



Mitigating Adverse Selection in Insurance 615

OK

Application Adverse Selection and Compulsory Insurance 616



15.2 Moral Hazard 617

Right moral hazard...

An Example of Extreme Moral Hazard 618

OK

Figure It Out 15.2 620

OKOK

Examples of Moral Hazard in Insurance Markets 620

Right moral hazard...

Moral Hazard outside Insurance Markets 621

Righto moral hazard...

Mitigating Moral Hazard 622

Rightyitty righto.

Application Usage-Based Auto Insurance 623



15.3 Asymmetrical Information in Principal-Agent Relationships 623

OK

Principal-Agent and Moral Hazard: An Example 624



The Principal-Agent Relationship as a Game 625



Freakonomics Yo-ho-ho . . . and Fair Treatment for All? 627

OK

More General Principal-Agent Relationships 628

OK

Theory and Data The Principal-Agent Problem in Residential Real Estate Transactions 628



Figure It Out 15.3 629



15.4 Signaling to Solve Asymmetric Information Problems 630

Yes we do more interesting things in our Game Theory courses and Notes.

The Classic Signaling Example: Education 630

Yes righto.

Other Signals 634

Tons of other signals bro.

Application Advertising as a Signal of Quality 635

OK

Figure It Out 15.4 635



15.5 Conclusion 636

OK solid chapter I would surmise.

Summary 636

OK

Review Questions 636

OK

Problems 637

OKOK

16 Externalities And Public Goods 643

OK

16.1 Externalities 645

OK

Why Things Go Wrong: The Economic Inefficiencies From Externalities 645

OK

Negative Externalities: Too Much Of A Bad Thing 645

OK

Figure It Out 16.1 647

OK

Positive Externalities: Not Enough Of A Good Thing 648

OK

Theory And Data The Positive Externality Of LoJack 650

OK

16.2 Fixing Externalities 651

OK

The Efficient Level Of Pollution 652

OK

Using Prices To Fix Externalities 653

OK

Application Reducing Spam 655

OK

Figure It Out 16.2 656

OK

Application Would Higher Driving Taxes Make Us Happier Drivers? 657

OK

Quantity Mechanisms To Reduce Externalities 658

OK

Price-Based Versus Quantity-Based Interventions With Uncertainty 659

OK

A Market-Oriented Approach To Reducing Externalities: Tradable Permits Markets 662

OK

Figure It Out 16.3 664

OK

16.3 Further Topics In Externalities And Their Remedies 666

OK

The Tragedy Of The Commons 666

OK

The Coase Theorem: Free Markets Fixing Externalities On Their Own 667

OK

Figure It Out 16.4 669

OK

Application The Tragedy Of The Commons Meets The Coase Theorem In The Texas Oil

OK

Fields 670

OK

The Coase Theorem And Tradable Permits Markets 670

OK

16.4 Public Goods 671

OK

The Optimal Level Of Public Goods 673

OK

Figure It Out 16.5 674

OK

Solving The Free-Rider Problem 675

OK

Freakonomics Is Fire Protection A Public Good? 677

OK

16.5 Conclusion 678

OK

Summary 678

OK

Review Questions 679

OK

Problems 679

OK

17 Behavioral And Experimental Economics 685

OK

17.1 When Human Beings Fail To Act The Way Economic Models Predict 687

OK

Systematic Bias 1: Overconfidence 687

OK

Systematic Bias 2: Self-Control Problems And Hyperbolic Discounting 688

OK

Systematic Bias 3: Falling Prey To Framing 690

OK

Systematic Bias 4: Paying Attention To Sunk Costs 692

OK

Application Sunk Cost Bias and the Breakdown of the Housing Market 693

OK

Systematic Bias 5: Generosity and Selfl essness 694

OK

17.2 Does Behavioral Economics Mean Everything We've Learned Is Useless? 695

OK

17.3 Testing Economic Theories with Data: Experimental Economics 696

OK

Lab Experiments 697

OK

Freakonomics Going to the Ends of the World [Literally] to Test Economic Theory 698

OK

Natural Experiments and Field Experiments 699

OK

17.4 Conclusions and the Future of Microeconomics 700

OK

Summary 701

OK

Review Questions 701

OK

Problems 701

OK

Math Review Appendix MR-1

OK

Solutions to Review Questions SA-1

OK

Solutions to Select End-of-Chapter Problems SB-1

OK

Glossary G-1

OK

References R-1

OK

Index I-1

OK
